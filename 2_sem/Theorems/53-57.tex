\documentclass[../main.tex]{subfiles}
\graphicspath{{\subfix{../Images/}}}
\begin{document}


\subsection{Признак Коши сходимости положительных рядов (pro)}
    \subsubsection*{Формулировка}
        
            Пусть $a_n \geq 0$, $k = \overline{\lim\limits_{n \rightarrow \infty}} \sqrt[n]{a_n}$
            
            \begin{enumerate}
            
                \item $k > 1$, значит $\sum a_n$ ~--- расходится
                
                \item $k < 1$, значит $\sum a_n$ ~--- сходится
                
            \end{enumerate}
            
        \subsubsection*{Доказательство}
        
            \begin{enumerate}
            
                \item Пусть $k > 1$, тогда для бесконечного числа номеров $\sqrt[n]{a_n} > 1$, а значит $a_n > 1$, значит $a_n$ не стремится к $0$, и поэтому ряд расходится.
                
                \item Пусть $k < 1$. Обозначим за $\varepsilon = \dfrac{1 - k}{2} > 0$, $q = \dfrac{1 + k}{2}$. По свойствам верхнего предела существует такое $N$, что для всех $n > N$ выполняется неравенство 
                
                    $$\sqrt[n]{a_n} < k + \varepsilon = \dfrac{1 + k}{2} = q \in (0, 1)$$
                    
                    Тогда $a_n < q^n$ при всех $n > N$, и ряд $\sum\limits^{\infty}_{k = 1} a_k$ сходится по признаку сравнения со сходящимся рядом $\sum\limits^{\infty}_{k = 1} q^k$
                
            \end{enumerate}
\newpage


\subsection{Признак Даламбера сходимости положительных рядов}
\subsubsection*{Формулировка}
        
            Пусть $a_n \geq 0$, $D_n = \dfrac{a_{n + 1}}{a_n}$
            
            $\underline{light}$
            
            \begin{enumerate}
            
                \item $\exists q < 1$ начиная с некоторого места $D_n \leq q$, значит $\sum a_n$ сходится
                
                \item $D_n \geq 1$ начиная с некоторого места, значит $\sum a_n$ расходится
                
            \end{enumerate}
            
            $\underline{pro}$
            
            Пусть $\exists \lim \dfrac{a_{n + 1}}{a_n} = D$
            
            \begin{enumerate}
            
                \item $D < 1$, значит $\sum a_n$ сходится
                
                \item $D > 1$, значит $\sum a_n$ расходится
                
            \end{enumerate}
    
        \subsubsection*{Доказательство}
        
            \underline{light}
            
            \begin{enumerate}
            
                \item 
                
                    $\dfrac{a_{N + 1}}{a_N} < q$
                    
                    $\dfrac{a_{N + 2}}{a_{N + 1}} < q$
                    
                    $\ldots$
                    
                    $\dfrac{a_{N + k}}{a_{N + k - 1}} < q$
                    
                    $a_{N + k} < q^k \cdot a_{N_0}$ ~--- сходится
                    
                    Значит $a_n$ сходится
                    
                \item
                
                    $a_{N_0 + k} \geq a_{N_0} > 0$, значит $a_k$ не стремится $0$ ~--- расходится
                
            \end{enumerate}
            
            \underline{pro}
            
            \begin{enumerate}
            
                \item $\lim \dfrac{a_{n + 1}}{a_n} = D$, значит НСНМ $\dfrac{a_{n + 1}}{a_n} < q$, значит $\sum a_n$ сходится
                
                \item $\lim \dfrac{a_{n + 1}}{a_n} = D > 1$, значит НСНМ $\dfrac{a_{n + 1}}{a_n} > 1$, значит $\sum a_n$ расходится
                
            \end{enumerate}
\newpage


\subsection{Признак Раабе сходимости положительных рядов}
\subsubsection*{Лемма}
        
            \subsubsection*{Формулировка}
            
                Пусть $a_n$, $b_n > 0$ и $\dfrac{a_{n + 1}}{a_n} < \dfrac{b_{n + 1}}{b_n}$ НСНМ. Тогда
                
                $b_n$ ~--- сходится, значит и $a_n$ сходится
                
                или
                
                $a_n$ ~--- расходится, значит и $b_n$ расходится.
                
            \subsubsection*{Доказательство}
            
                Будем считать "НСНМ" как "1"
                
                $a_2 < a_1 \dfrac{b_2}{b_1}$
                
                $a_3 < a_2 \dfrac{b_3}{b_2}$
                
                $\ldots$
                
                $a_n < a_{n - 1} \dfrac{b_n}{b_{n - 1}}$, значит $a_n < \dfrac{a_1}{b_1} b_n$, т.е. $a_n < c \cdot b_n$
        
        \subsubsection*{Теорема}
        
            \subsubsection*{Формулировка}
            
                $a_n > 0$, тогда если 
                
                $n \cdot \left( \dfrac{a_n}{a_{n + 1}} - 1 \right) \geq r > 1$ (НСНМ), тогда $\sum a_n$ ~--- сходится
                
                $n \cdot \left( \dfrac{a_n}{a_{n + 1}} - 1 \right) \leq 1$ (НСНМ), тогда $\sum a_n$ ~--- расходится
                
            \subsubsection*{Доказательство}
            
                \begin{enumerate}
                
                    \item $n \cdot \left( \dfrac{a_n}{a_{n + 1}} - 1 \right) \geq r \Rightarrow \dfrac{a_n}{a_{n + 1}} \geq 1 + \dfrac{r}{n}$
                    
                    Пусть $1 < s < r$, $b_n := \dfrac{1}{n^s}$
                    
                    Итак, НСНМ $\dfrac{a_{n + 1}}{a_n} < \dfrac{b_{n + 1}}{b_n}$
                    
                    $\sum b_n = \sum \dfrac{1}{n^s}$ ~--- сходится, значит $\sum a_n$ ~--- сходится
                    
                    \item $n \left( \dfrac{a_n}{a_{n + 1}} - 1 \right) \leq 1$, $\dfrac{a_n}{a_{n + 1}} \leq \dfrac{n + 1}{n} = \dfrac{\frac{1}{n}}{\frac{1}{n + 1}}$
                    
                    $\dfrac{\frac{1}{n + 1}}{\frac{1}{n}} \leq \dfrac{a_{n + 1}}{a_n}$, $\sum \frac{1}{n}$ ~--- расходится, значит и $\sum a_n$ ~--- расходится
                    
                \end{enumerate}
                
\newpage


\subsection{Интегральный признак Коши сходимости числовых рядов}
 \subsubsection*{Формулировка}
        
            Пусть $f : [1, +\infty) \rightarrow \mathbb{R}$, непрерывна, $\geq 0$, монотонна
            
            Тогда $\sum\limits^{+\infty}_{k = 1} f(k)$ и $\int\limits^{+\infty}_1 f(x) dx$ ~--- сходится или расходится одновременно. Содержательный случай $f$ ~--- убывает и $f(1) > 0$
            
        \subsubsection*{Доказательство}
        
            \begin{itemize}
            
            \item Ряд сходится, значит $S^{(f)}_n$ ~--- ограничена сверху
            
                Тогда $\Phi(A) = \int\limits^A_1 f(x) dx$ ~--- ограничена сверху
            
                $S^{(f)}_n \leq S$
            
                $\Phi(A) < \Phi([A] + 1) = \int\limits^{[A] + 1}_1 f(x) dx = \sum\limits^{[A]}_{k = 1} \int\limits^{k + 1}_k f(x)dx \leq \sum \int\limits^{k + 1}_k f(k) dx = \sum\limits^{[A]}_{k = 1} f(k) \leq S$
            
            \item Интеграл сходится, значит и ряд сходится
            
                $\Phi(A) \leq S$
            
                Проверим, что $S_n \leq S + f(1)$
            
                $S_n = \sum\limits^n_{k = 1} f(k) = f(1) + \sum\limits^n_{k = 2} \int\limits^k_{k - 1} f(k) dx \leq f(1) + \sum\limits^n_{k = 2} \int\limits^k_{k - 1} f(x) dx = f(1) + \int\limits^n_1 f(x) dx \leq f(1) + S$
                
            \end{itemize}
\newpage


\subsection{Признак Лейбница}
 \subsubsection*{Формулировка}
        
            Пусть $a_1 \geq a_2 \geq a_3 \geq \ldots \geq 0$, $a_n \rightarrow 0$. Тогда $\sum\limits^{+\infty}_{k = 1} (-1)^{k - 1} a_k$ ~--- сходится
            
        \subsubsection*{Доказательство}
        
            $S_{2n} = (a_1 - a_2) + (a_3 - a_4) + \ldots + (a_{2n - 1} - a_{2n})$
            
            $S_{2n + 2} = S_{2n} + a_{2n + 1} - a_{2n + 2} \geq S_{2n}$
            
            $S_{2n} \leq a_1$, $S_{2n} = a_1 - (a_2 - a_3) - \ldots (a_{2n - 2} - a_{2n - 1}) - a_{2n}$
            
            $S_{2n + 1} = S_{2n} + a_{2n + 1}$, итого $S$ ~--- ограничено, значит ряд сходится
\newpage


\end{document}