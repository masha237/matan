\documentclass[../main.tex]{subfiles}
\graphicspath{{\subfix{../Images/}}}
\begin{document}


\subsection{Неравенство Гельдера для сумм}

Пусть $p > 1$, $\frac{1}{p} + \frac{1}{q} = 1$\\\\
$q = \frac{p}{p - 1}$\\\\
$a_i, b_i > 0$ для всех $i = 1..n$\\\\
Тогда $\sum\limits^n_{i = 1} a_i b_i \leq (\sum a_i^p)^{\frac{1}{p}} (\sum b_i^q)^{\frac{1}{q}}$\\\\
Если $(a_1^p, a_2^p, \ldots, a_n^p) \parallel (b_1^q, b_2^q, \ldots, b_n^q)$ ~--- равенство\\\\\\
\textbf{Доказательство}\\\\
$x^p$ ~--- строго выпукла при $p > 1$ и $x > 0$\\\\
$(x^p)'' = p(p - 1)x^{p - 2} > 0$\\\\
По неравенству Йенсена $\left(\sum\limits_{i = 1}^n \alpha_i x_i \right)^p \leq \sum\limits_{i = 1}^n \alpha_i x_i^p$\\\\
$\alpha_i := \dfrac{b_i^q}{\sum b_i^q}$\\\\
$\alpha_i > 0$, $\sum \alpha_i = 1$\\\\
Выберем такие $x_i$, что\\\\
$\alpha_i \cdot x_i = a_i \cdot b_i$\\\\
$x_i = \dfrac{a_i b_i}{\alpha_i} = \dfrac{a_i b_i}{b_i^q} \sum\limits_{j = 1}^n b_j^q = a_i b_i^{1 - q} \sum\limits_{j = 1}^n b_j^q = a_i b_i^{1 - \frac{p}{p - 1}} \sum\limits_{j = 1}^n b_j^q = a_i b_i^{\frac{p - 1 - p}{p - 1}} \sum\limits_{j = 1}^n b_j^q = a_i \cdot b_i^{-\frac{1}{p - 1}} \sum\limits_{j = 1}^n b_j^q$\\\\
Тогда $\alpha_i x_i = a_i b_i$\\\\
$(\sum\limits_{i = 1}^n \alpha_i x_i)^p = (\sum\limits_{i = 1}^n a_i b_i)^p$\\\\
Тогда $\alpha_i x_i^p = a_i^p(\sum\limits_{j = 1}^n b_j^q)^{p - 1}$\\\\
Тогда $\sum\limits_{i = 1}^n \alpha_i x_i^p = (\sum\limits_{i = 1}^n a_i^p)(\sum\limits_{j = 1}^n b_j^q)^{p - 1} = (\sum\limits_{i = 1}^n a_i^p)(\sum\limits_{j = 1}^n b_j^q)^{\frac{p}{q}}$\\\\
Тогда $(\sum_{i = 1}^n a_i b_i)^p \leq (\sum\limits_{i = 1}^n a_i^p)(\sum\limits_{j = 1}^n b_j^q)^{\frac{p}{q}}$\\\\
Возведём в степень $\frac{1}{p}$ и получим исходное неравенство

\newpage


\subsection{Неравенство Гельдера для интегралов}
Пусть $\frac{1}{p} + \frac{1}{q} = 1$, $p > 1$\\\\
Пусть также $f$, $g \in C[a, b]$ и $f, g \geq 0$ на $[a, b]$. Тогда\\\\
$\int\limits^b_a fg \leq \left(\int\limits^b_a f^p\right)^{\frac{1}{p}} \left(\int\limits^b_a g^q\right)^{\frac{1}{q}}$\\\\\\
\textbf{Доказательство}\\\\
Делим $[a, b]$ на $n$ равных частей\\\\
$x_k = a + k \cdot \frac{b - a}{n}$\\\\
$\Delta x_k = x_k - x_{k - 1} = \frac{b - a}{n}$\\\\
$\xi_k := x_k$\\\\
$a_k := |f(x_k)|(\Delta x_k)^{\frac{1}{p}}$\\\\
$b_k := |g(x_k)|(\Delta x_k)^{\frac{1}{q}}$\\\\
$a_k \cdot b_k = |f(x_k)g(x_k)| \cdot \Delta x_k$\\\\
$\sum\limits_{k = 1}^n |f(x_k) g(x_k)| \Delta x_k \leq (\sum |f(x_k)|^p \Delta x_k)^{\frac{1}{p}}(\sum|g(x_k)|^q \Delta x_k)^{\frac{1}{q}}$\\\\
Из неравенства Гёльдера для сумм\\\\
$\int\limits^b_a |f(x) g(x)| dx \leq (\int\limits^b_a |f|^p)^{\frac{1}{p}}(\int\limits^b_a |g|^q)^{\frac{1}{q}}$

\newpage


\subsection{Неравенство Минковского}
Пусть $p \geq 1$\\\\
Тогда $\left( \sum\limits_{i = 1}^n |a_i + b_i|^p\right)^{\frac{1}{p}} \leq \left( \sum |a_i|^p \right)^{\frac{1}{p}} + \left( \sum |b_i|^p \right)^{\frac{1}{p}}$\\\\
$a_i, b_i \in \mathbb{R}$\\\\\\
\textbf{Замечания}\\\\
Неравенство Минковского означает, что $(a_1, a_2, \ldots, a_n) \mapsto \left( \sum |a_i|^p \right)^{\frac{1}{p}}$ является нормой\\\\\\
\textbf{Доказательство}\\\\
При $p = 1$ ~--- очевидно\\\\
При $p > 1$ ~--- применим Гёльдера\\\\
Пусть $a_i, b_i > 0$\\\\
$\sum |a_i| |a_i + b_i|^{p - 1} \leq \left( \sum |a_i|^p \right)^{\frac{1}{p}} \left( \sum |a_i + b_i|^p \right)^{\frac{1}{q}}$\\\\
$\sum |b_i| |a_i + b_i|^{p - 1} \leq \left( \sum |b_i|^p \right)^{\frac{1}{p}} \left( \sum |a_i + b_i|^p \right)^{\frac{1}{q}}$\\\\
$\sum |a_i + b_i|^p \leq \sum ( |a_i| + |b_i| ) |a_i + b_i|^{p - 1} \leq \left( \left( \sum |a_i|^p \right)^{\frac{1}{p}} + \left( \sum |b_i|^p \right)^{\frac{1}{p}} \right) \left(\sum |a_i + b_i|^p \right)^{\frac{1}{q}}$\\\\
$\left( \sum |a_i + b_i|^p \right)^{\frac{1}{p}} \leq \ldots \leq \left( \sum |a_i|^p \right)^{\frac{1}{p}} + \left( \sum |b_i|^p \right)^{\frac{1}{p}}$

\newpage


\subsection{Простейшие свойства несобственного интеграла}

1. Критерий Больцано-Коши сходимости несобственного интеграла\\\\
Сходимость интеграла $\int\limits^{\rightarrow b}_a f$ равносильна\\\\
$\forall \varepsilon > 0 : \exists \Delta \in (a, b) : \forall B_1, B_2 : \Delta < B_1 < B_2 < b : \left| \int\limits^{B_2}_{B_1} f \right| < \varepsilon$\\\\\\
2. $f$ ~--- допустима на $[a, b)$ и $C \in (a, b)$. Тогда \\\\
$\int\limits^{\rightarrow b}_a f$ и $\int\limits^{\rightarrow b}_c f$ сходятся и расходятся одновременно, и при этом в случае сходимости $\int\limits^{\rightarrow b}_a = \int\limits^c_a + \int\limits^{\rightarrow b}_c$ \\\\\\\
3. Пусть $f$, $g$ ~--- допустимы на $[a, b)$, а также \\\\
$\int\limits^{\rightarrow b}_a f$ и $\int\limits^{\rightarrow b}_a g$ сходятся. Пусть $\lambda \in \mathbb{R}$, тогда \\\\
$\lambda f$ и $f \pm g$ ~--- допустимые функции на $[a, b)$ и \\\\
$\int\limits^{\rightarrow b}_a \lambda f = \lambda \int\limits^{\rightarrow b}_a f$ и $\int\limits^{\rightarrow b}_a f \pm g = \int\limits^{\rightarrow b}_a f \pm \int\limits^{\rightarrow b}_a g$\\\\\\
4. Пусть $\int\limits^{\rightarrow b}_a f$ и $\int\limits^{\rightarrow b}_a g$ существуют в $\overline{\mathbb{R}}$, $f \leq g$ на $[a, b)$ Тогда \\\\
$\int\limits^{\rightarrow b}_a f \leq \int\limits^{\rightarrow b}_a g$ \\\\\\
5. Пусть $f$, $g$ ~--- дифференцируемы на $[a, b)$, $f'$, $g'$ ~--- допустимы на $[a, b)$. Тогда (при существовании двух из трёх пределов) \\\\
$\int\limits^{\rightarrow b}_a f g' = fg \bigg|^{\rightarrow b}_a - \int\limits^{\rightarrow b}_a f'g$ \\\\\\
6. Пусть $\varphi : [\alpha, \beta) \rightarrow \langle A, B \rangle$, $\varphi \in C^1 \left( [\alpha, \beta) \right)$, $f \in C \left( \langle A, B \rangle \right)$. Пусть также существует $\varphi(\beta - 0) \in \overline{\mathbb{R}}$. Тогда \\\\
$\int\limits^{\rightarrow b}_a (f \circ \varphi) \cdot \varphi' = \int\limits^{\varphi(\beta - 0)}_{\varphi(\alpha)} f$
       
\newpage
\textbf{Доказательство}\\\\\
1. Положим $\Phi(A) = \int\limits^A_a f$. Сходимость интеграла равносильна сходимости $\Phi(A)$ при $A \rightarrow b - 0$. Воспользуемся критерием Больцано-Коши, а также учтём, что $\Phi(B) - \Phi(A) = \int\limits^B_a$ \\\\
$$\forall \varepsilon > 0 : \exists \Delta \in (a, b) : \forall B_1, B_2 : \Delta < B_1 < B_2 < b : \left| \Phi(B_2) - \Phi(B_1) \right| < \varepsilon$$\\\\\\
2. При всех $A \in (c, b)$ согласно аддитивности интеграла\\\\          
$$\int\limits^A_a f = \int\limits^c_a f + \int\limits^A_c f$$\\\\\\
3. Аналогично предыдущему пункту возьмём такие $A$ и согласно линейности интеграла\\\\
$$\int\limits^A_a (\alpha f + \beta g) = \alpha \int\limits^A_a f + \beta \int\limits^A_a g$$\\\\\\
4. Также выберем $A$ и очевидно, что \\\\
$$\int\limits^A_a f \leq \int\limits^A_a g$$\\\\\
5.Устремим $A$ к $\rightarrow b$ \\\\
$$\int\limits^A_a fg' = fg \bigg|^A_a - \int\limits^A_a f'g$$\\\\\\
6. Кохась сказал, что без доказательства. На экзамене отвечаем ему то же самое

\newpage


\subsection{Признаки сравнения сходимости несобственного интеграла}
        
            \begin{enumerate}
            
                \item Пусть $f$ ~--- допустима на $[a, b)$, $f \geq 0$, $\Phi(B) = \int\limits^B_a f$. Тогда сходимость $\int\limits^b_a f$ равносильна ограниченности функции $\Phi$ (это не признак сравнения)
                
                \item Признаки сравнения
                
                    Пусть $f$, $g > 0$ и допустимы на $[a, b)$
                    
                    \begin{itemize}
                    
                        \item
                        
                            Если $f \leq g$ на $[a, b)$
                            
                            \begin{enumerate}
                            
                                \item $\int\limits^b_a g$ ~--- сходится, значит и $\int\limits^b_a f$ ~--- сходится
                                
                                \item $\int\limits^b_a f$ ~--- расходится, значит и $\int\limits^b_a g$ ~--- расходится
                                
                            \end{enumerate}
                        
                        \item Пусть существует $\lim\limits_{x \rightarrow b - 0} \dfrac{f(x)}{g(x)} = l$
                        
                            Тогда
                            
                            \begin{enumerate}
                            
                                \item $\int\limits^b_a g$ ~--- сходится, значит и $\int\limits^b_a f$ сходится, если $l \in [0, +\infty)$
                                
                                \item $\int\limits^b_a f$ и $\int\limits^b_a g$ сходятся и расходятся одновременно, если $l \in (0, +\infty)$
                                
                            \end{enumerate}
                        
                    \end{itemize}
                
            \end{enumerate}
\newpage
\textbf{Доказательство}
        
            \begin{enumerate}
            
                \item Очевидно, что $\Phi$ ~--- монотонно возрастает, тогда существование $\lim\limits_{B \rightarrow b - 0} \Phi \Longleftrightarrow \Phi$ ~--- ограничена
                
                \item 
                
                    \begin{itemize}
                    
                        \item Пусть $\Phi(B) = \int\limits^B_a f$, $\psi(B) = \int\limits^B_a g$, тогда $\Phi$, $\psi$ ~--- монотонные 
                        
                            $\Phi(B) \leq \psi(B)$
                            
                            \begin{enumerate}
                            
                                \item $\int\limits^b_a g$ ~--- сходится, значит $G(B)$ ограничено сверху, значит $F(B)$ ограничено сверху, значит и $\int\limits^b_a f$ ~--- сходится
                                
                                \item $\int\limits^b_a f$ ~--- расходится, значит $F(B)$ неограничено сверху, значит и $G(B)$ неограничено, значит и $\int\limits^b_a g$ ~--- расходится
                                
                            \end{enumerate}
                        
                        \item 
                            
                            \begin{enumerate}
                            
                                \item Возьмём $L > l$. Тогда существует $c \in [a, b) : \forall x \in [c, b)$
                                
                                    $f(x) \leq L \cdot g(x)$ Заменим $\int\limits^b_a$ на $\int\limits^b_c$. Тогда $\int\limits^b_c g$ ~--- сходится, значит и $\int\limits^b_c L g$ ~--- сходится и $\int\limits^b_c f$ ~--- сходится
                                
                                \item Для $l > 0$ аналогично и $\lambda < l$ и по аналогии $\lim \dfrac{g}{f} = \dfrac{1}{l}$ и $\int\limits^b_a f$ ~--- сходится $\Rightarrow \int\limits^b_a g$ ~--- сходится
                                
                            \end{enumerate}
                        
                    \end{itemize}
                
            \end{enumerate}
    

\newpage


\end{document}