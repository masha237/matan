\documentclass[../main.tex]{subfiles}
\graphicspath{{\subfix{../Images/}}}
\begin{document}

\subsection{Теорема Лагранжа для векторнозначных функций}
% \href{https://youtu.be/NVkI1RvNsUo?t=6845}{Запись}
\begin{theorem}
$F: [a, b] \rightarrow \R^l$, непрерывная на $[a, b]$, дифференцируемая на $(a, b)$
Тогда $\exists c \in (a, b):$
$$ | F(b) - F(a)| \leqslant (b - a) |F'(c)| $$
\end{theorem}
\begin{proof}
Рассмотрим следующую функцию: $$\varphi(t):=\scalar{ F(b) - F(a), F(t) - F(a)}, \ t \in [a, b]$$
    $$\varphi(a) = 0, \ \varphi(b) = |F(b) - F(a)|^2$$
    $$\varphi'(t) = \scalar{ F(b) - F(a), F'(t)}$$
    Функция $\varphi$ непрерывна на $[a, b]$ и дифференцируема на $(a, b)$, применим т. Лагранжа:
    $$\varphi(b)-\varphi(a) = \varphi'(c) \cdot (b - a)$$
    $$| F(b) - F(a)|^2 = \scalar{ F(b) - F(a), F'(c)} \cdot (b - a) \leqslant \abs{F(b) - F(a)} \cdot \abs{F'(c)}(b - a)$$
    $$\abs{F(b) - F(a)} \leqslant (b - a) \abs{F'(c)}$$
\end{proof}
\newpage

\subsection{Экстремальное свойство градиента}
% \href{https://youtu.be/NVkI1RvNsUo?t=8864}{Запись}
\begin{theorem}
$$f : E \subset \R^m \rightarrow \R, \ a \in \Interior E, \ \vec{l} = \frac{\nabla f(a)}{|\nabla f(a)|},$$
тогда $\vec{l}$ --- направление наискорейшего возрастания функции, т.е. $$\forall h, \ \abs{h} = 1 \qquad
-\frac{\partial f}{\partial l}(a) \leqslant \frac{\partial f}{\partial h}(a) \leqslant \frac{\partial f}{\partial l}(a)$$
\end{theorem}
\begin{proof}
    $$\frac{\partial f}{\partial h}(a) = \scalar{ \nabla f(a), h} \leqslant \abs{\nabla f(a)} \cdot \abs{h} = \scalar{ \nabla f(a), l } = \frac{\partial f}{\partial l}$$
    Ан-но для неравенства с минусом
\end{proof}
\newpage


\end{document}