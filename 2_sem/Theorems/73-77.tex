\documentclass[../main.tex]{subfiles}
\graphicspath{{\subfix{../Images/}}}
\begin{document}
\subsection{Формула Стирлинга для гамма-функции}
\begin{theorem}
    $$\Gamma(x + 1) \sim \sqrt{2\pi x} x^x e^{-x}, \ x \rightarrow + \infty$$
\end{theorem}
\begin{proof}
$$\Gamma(x + 1) = \int_0^{+ \infty} t^x e^{-t} dt \underset{t = yx}{=} \int_0^{+ \infty} (yx)^x e^{-yx} x dy = x^{x + 1} \int_0^{+ \infty} e^ {x(\ln y - y)} dy$$
Максимум функции $\ln y - y$ достигается при $y = 1$. Воспользуемся замечанием к теореме Лапласа. Для этого перейдем к его обозначениям:
$A = x, \ a = 1, \ \varphi''(a) = -1, \ L = 1, \ \varphi(a) = -1$. Получим:
$$\Gamma (x + 1) \sim \sqrt{\frac{2 \pi}{A}} \frac{L}{\sqrt{\abs{\varphi''(a)}}} e^{A \varphi(a)} = x ^ {x + 1} \sqrt{2 \pi} \frac{1}{\sqrt{1}}$$
\end{proof}
\newpage

\subsection{Теоерма Вейерштрасса о многочленах}
\begin{theorem}
    $$\forall f \in C[a, b] \quad \exists \text{~послед. мн-нов~} P_n(x): \quad \forall x \in [a, b] \quad P_n(x) \rightarrow f(x)$$
\end{theorem}
\begin{proof}
Рассмотрим следующую последовательность многочленов: % очевидно же?
$$P_n(x) = \frac {1}{b - a} \sqrt{\frac{n}{2 \pi}} \int_a^b f(t) \left( 1 - \frac{(x - t)^2}{(b - a)^2} \right)^n dt$$
$$\int_a^b f(t) \left( 1 - \frac{(x - t)^2}{(b - a)^2} \right)^n dt = 
\int_a^b f(t) e^{n \ln\left(1 - \frac{(x - t)^2}{(b - a)^2}\right)}$$
Обозначим логарифм за $\varphi(t)$. Функция $\varphi$ имеет максимум. Воспользуемся методом Лапласа и получим:
$$\int_a^b f(t) \left( 1 - \frac{(x - t)^2}{(b - a)^2} \right)^n dt \underset{n \rightarrow + \infty}{\sim} \sqrt{\frac{2 \pi}{n}} f(x) (b - a)$$
$$P_n(x) \underset{n \rightarrow + \infty}{\sim} f(x)$$
\end{proof}
\newpage

\subsection{Лемма об оценке нормы линейного оператора}
\begin{lemma}
 $A: \R^m \rightarrow \R^l$, лин. оператор $A=(a_{ij})$ \\
 $\forall x \in \R^m \qquad |Ax|^2 \leqslant C_A |x|$, где $C_A = (\sum a^2_{i j})^{\frac{1}{2}}$
\end{lemma}
\begin{proof}
$$ |Ax|^2 = \sum_{i=1}^l \left( \sum_{j = 1}^m a_{i j} x_j \right)^2  \underset{\text{КБШ}}{\leqslant} \sum_{i = 1}^l \left( \left( \sum_{j = 1}^m a_{i j}^2 \right )\left( \sum_{j = 1}^m x_j^2 \right )\right) = \sum x^2_j \cdot \sum_{i, j} a_{i j}^2 = \sum x^2_j \cdot C^2_A
$$
\end{proof}
\begin{consequence}
Линейное отображение всюду непрерывно
\end{consequence}
\newpage

\subsection{Дифференцирование композиции}
\begin{theorem}
$F : E \subset \R^m \rightarrow \R^l, \ a \in \mathcal{Int} E, \ G : I  \subset \R^l \rightarrow \R^n, \ b = F(a), \ b \in \Interior I, \ F $~--- дифф. в $a, \ G$~--- дифф. в $F(a)$, тогда $G \circ F$~--- дифф. и
$$(G \circ F)'(a)= G'(F(a)) \cdot F'(a)$$
\end{theorem}
\begin{proof}
$$ F(a + h) = F(a) + F'(a) h + \alpha(h) \abs{h}$$
$$ F(b + k) = G(b) + G'(b) k + \beta(k) \abs{k}$$
Подставим $k=k(h) = F'(a)h + \alpha(h)\abs(h),$ где $\gamma(h) = G'(b) \alpha(h) + \beta(k) \frac{\abs{k}}{\abs{h}}$. Докажем, что $\gamma(h)$~--- бесконечно малое при $h \rightarrow 0$
$$|G'(b) \alpha(h)| \leqslant C_{G'(b)} \cdot \abs{\alpha(h)} \text{~--- б.м.}$$
$$\begin{array}{c}
    \abs{k} = \abs{F'(a)h + \alpha(h)\abs{h}} \leqslant \abs{f'(a)h} + \abs{\alpha(h)} \abs{h} \leqslant \underset{\text{огр.}}{\left( C_{F'(a)} + \abs{\alpha(h)} \right)} \abs{h} \Rightarrow \\ \frac{\abs{k}}{\abs{h}} \text{~--- огр.}, \ \beta(k) \text{~--- б.м.}
\end{array}$$
\end{proof}
\newpage
\subsection{Дифференцирование <<произведений>>}
\begin{lemma}
$F, G: E \subset \R^m \rightarrow \R^l, \ a \in \Interior E, \ \lambda: E \rightarrow \R, \ F, G, \lambda$ ~--- дифференцируемы в $a$, \\
Тогда: $\lambda F, \ \scalar{F, G}$~--- дифференцируемы в $a$ и
\begin{enumerate}
    \item
    $(\lambda F)'(a)h = (\lambda'(a)h)F(a) + \lambda(a) F'(a)h$
    \item 
    $(\scalar{F, G})'(a)h= \scalar{F'(a)h, G(a)} + \scalar{F(a), G'(a)h}$
\end{enumerate}
\end{lemma}
\begin{proof}
~

    \begin{enumerate}
        \item 
        Рассмотрим координатную функцию ($l=1$, \ $F \leftrightarrow f$):
        $\lambda f (a + h) - \lambda f(a) = (\lambda(a) + \lambda'(a)h + \alpha(h) \abs{h})(f(a) + f'(a)h + \beta(h) \abs{h}) - \lambda(a)f(a) = \lambda'(a)h \cdot f(a) + \lambda(a) f'(a)h + \abs{h} (\alpha(h)f(a) + \cdots) + \alpha'(a)h \cdot f'(a)h = \lambda'(a)h \cdot f(a) + \lambda(a) f'(a)h$ \\
        Общий случай получим по принципу покоординатной сходимости
        \item 
        $\scalar{F, G}'(a)h = \left( \sum_{i = 1}^l f_i g_i \right)' (a) h = \sum (f_i g_i)'(a)h = \sum f_i'(a)h \cdot g_i + \sum f_i\cdot g_i'(a)h = \scalar{F'(a)h, G(a)} + \scalar{F(a), G'(a)h}$
    \end{enumerate}
\end{proof}
\newpage

\end{document}