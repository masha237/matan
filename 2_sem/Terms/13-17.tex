\documentclass[../main.tex]{subfiles}
\graphicspath{{\subfix{../Images/}}}
\begin{document}


\subsection{Плотность аддитивной функции промежутка}
$f: <a,b> \rightarrow \mathbf{R}$, $\phi : Segm<a,b> \rightarrow \mathbf{R}$ \\ 
$\phi$ - аддитивная функция промежутка. \\
$f$ - плотность $\phi$, если: \\ 
$$\forall \Delta \in Segm<a, b> : \inf_{x \in \Delta} f(x) \cdot l(\Delta) \leq \phi (\Delta) \leq \sup_{x \in \Delta} f(x) \cdot l(\Delta)$$


\subsection{Выпуклая функция}
$f:<a,b> -> R$ выпуклая, если
$$$$
$\forall x, y \in <a, b> \alpha \in [0,1] f(\alpha x + (1 - \alpha)y) \leq \alpha f(x) + (1 - \alpha)f(y)$
$$$$
График f ниже любой хорды.


\subsection{Выпуклое множество в $R^m$}
$A \subset R^n$ -- выпуклое, если $\forall x, y \in A : \ [x, y] \in A$
$$$$
$[x, y] = \{x + t(y - x)| t \in [0, 1]\}$


\subsection{Надграфик}
$\{(x,y) \ x \in <a, b> \ y \geq f(x)\}$

\newpage

\lhead{Акулов}
\subsection{Опорная прямая}
Прямая в $\mathbb{R^2}$, проходящая через $(x, f(x))$, такая что график $f(x)$ лежит в одной полуплоскости относительно этой прямой. \\
Не путать с касательной. Например для $f(x) = x$ прямая $y = x$ -- опорная, а для $f(x) = |x|$ прямые $y = kx \ \forall k \in [-1; 1]$ -- опорные. \\
Аналогичное определение и для фигур в $\mathbb{R^2}$.
\newpage

\end{document}