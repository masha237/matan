\documentclass[../main.tex]{subfiles}
\graphicspath{{\subfix{../Images/}}}
\begin{document}

\subsection{$o(h)$ при $h\rightarrow0$}
https://youtu.be/wWtKfrt\_ZQY?t=1h40m50s
\begin{gather*}
    \varphi: E\subset \R^m \to \R^l \\
    \0 \in \operatorname{Int} E \\
    \varphi(h)=o(h)\text{ при } h\to 0 \Leftrightarrow
    \frac{\varphi(h)}{|h|} \text{ б.м. при } h\to 0 \\
    \text{По-другому: }\exists \alpha(h):E\to\R^l \text{ б.м. при } h\to 0: \ \varphi(h)=|h|\alpha(h)
\end{gather*}

\subsection{Отображение, дифференцируемое в точке}\label{subsec:1.49}
https://youtu.be/wWtKfrt\_ZQY?t=1h46m03s
\begin{gather*}
    F: E\subset \R^m \to \R^l\\
    a \in \operatorname{Int}E \\
    F \text{ дифференцируемо в точке }a \Leftrightarrow
    \exists \text{ лин.оп.} L:\R^m \to \R^l,
    \exists \alpha(h):E\to\R^l \text{ б.м. при } h\to 0: \\
    F(a + h) = F(a) + Lh +|h|\alpha(h)\\
    F(a + h) = F(a) + Lh +o(h)\\
    x \overset{\text{def}}{=} a + h,\quad \varphi(x):E\to\R^l \text{ б.м. при } x \to a \\
    F(x) = F(a) + L(x-a) + |x-a| \varphi(x)
\end{gather*}

\subsection{Производный оператор, матрица Якоби, дифференциал}
https://youtu.be/wWtKfrt\_ZQY?t=1h54m42s

Оператор $L$ из~\ref{subsec:1.49}~---\textbf{производный оператор} отображения $F$ в точке $a$. Обозначается $F'(a)$.

Матрица $F'(a)$~--- \textbf{матрица Якоби} отображения $F$ в точке $a$.

$F'(a)h$~--- \textbf{дифференциал} отображения $F$ в точке $a$.
Под ним понимают:
\begin{itemize}
    \item Производный оператор, т.е. линейное отображение $h \mapsto F'(a) h$
    \item Отображение $E\times \R^m \to \R^l \quad (x, h) \mapsto F'(x) \cdot h$
\end{itemize}

\subsection{Теорема о двойном и повторном пределах}
https://youtu.be/wWtKfrt\_ZQY?t=1h02m28s а также см.~\ref{subsec:1.44}

$D_1, D_2, a, b, f, \varphi$ как в ~\ref{subsec:1.44}

\begin{equation*}
    \begin{cases}
        \exists \lim\limits_{\underset{y \to b}{x \to a}}f(x,y)=A \in \bar{\R} \\
        \forall x \in D_1 \setminus \{a\}\ \exists \varphi(x)=\lim\limits_{y \to b} f(x,y)
    \end{cases}
    \Rightarrow
    \exists \lim\limits_{x \to a} \varphi(x)
\end{equation*}
\begin{equation*}
    \begin{cases}
        \exists \lim\limits_{\underset{y \to b}{x \to a}}f(x,y)=A \in \bar{\R} \\
        \forall x \in D_1 \setminus \{a\}\ \exists \varphi(x)=\lim\limits_{y \to b} f(x,y)\\
        \forall y \in D_2 \setminus \{b\}\ \exists \psi(y)=\lim\limits_{x \to a} f(x,y)
    \end{cases}
    \Rightarrow
    \begin{cases}
        \exists \lim\limits_{x \to a} \varphi(x) \\
        \exists \lim\limits_{y \to b} \psi(y) \\
        \lim\limits_{x \to a} \varphi(x) = \lim\limits_{y \to b} \psi(y) = 
        \lim\limits_{\underset{y \to b}{x \to a}}f(x,y)
    \end{cases}
\end{equation*}

\subsection{Частные производные}
https://youtu.be/i6m1YzUm8mU?t=37s

\begin{gather*}
    f: E \subset \R^m \to \R \\
    a \in \operatorname{Int}E \\
    k \in \{1, \dots, m\} \\
   \varphi_k(t)\overset{\text{def}}{=}f(a_1, \dots, a_{k-1}, t, a_{k+1}, \dots, a_m),
   \quad  t \in U(a_k) 
\end{gather*}
Тогда частная проивзодная функции $f$ по $k$-й переменной в точке $a_k$~--- это
\begin{equation*}
   \varphi_k'(a_k)=\lim_{h \to 0}\frac{\varphi_k(a_k + h) - \varphi_k(a_k)}{h}
\end{equation*}

Обычно используются обозначения:
\begin{equation*}
    \frac{\partial f}{\partial x_k}(a), \quad D_k f(a), \quad f_k'(a), \quad f_{x_k}'(a)
\end{equation*}

Вообще можно не усложнять себе жизнь всякими промежуточными $\varphi_k(t)$ и записать сразу:
\begin{equation*}
    \frac{\partial f}{\partial x_k}(a)\overset{\text{def}}{=}
    \lim_{h \to 0}\frac{f(a_1, \dots, a_k + h, \dots, a_m)-f(a_1, \dots, a_k, \dots, a_m)}{h}
\end{equation*}
\newpage
\end{document}