\documentclass[../main.tex]{subfiles}
\graphicspath{{\subfix{../Images/}}}
\begin{document}


\subsection{Постоянная Эйлера}
https://youtu.be/SmopNVfVQss?t=1524

$1 + \frac{1}{2} + \frac{1}{3} .... + \frac{1}{n} = \ln{n} + \gamma + o(1)$

где $ \frac{1}{2} < \gamma < \frac{5}{8}$ - постоянная Эйлера


\subsection{Допустимая функция }
https://youtu.be/OKlsGDA9VYA?t=5014

$f : [a, b) \to \R \quad -\infty<a<b\leq+\infty$
(Можно $(a, b] -\infty\leq a<b<+\infty$)

$f$ \textbf{допустима}, если $f$ --- кусочно-непрерывна на $[a, A] \ \ \forall A\in (a, b)$


\subsection{Несобственный интеграл, сходимость, расходимость}
https://youtu.be/OKlsGDA9VYA?t=5260
    $$\Phi(A):=\int_a^A f dx$$

   Если  $ \exists\lim\limits_{A\to b-0} \Phi(A) \in \mathds{R}$ с чертой, то это \textbf{несобственный интеграл} который обозначается  $\int\limits_{a}^{\rightarrow b} fdx$.
          
    Если  $ \nexists\lim\limits_{A\to b-0} \Phi(A)$ то несобственный интеграл не существует, но если
    $\lim\limits_{A\to b-0} \Phi(A)\in \mathds{R}$,
    то несобственный интеграл \textbf{сходится}. (За сходиться на экзамене -балл)
     Если этот предел бесконечный или не существует, то несобственный интеграл \textbf{расходится}.
     
     
    Пример не КПК (2019 М3137)
    \begin{example}
    $$\int_1^{+\infty} \frac{1}{x^p} dx=\lim_{A\to\infty}\int_1^{A} \frac{1}{x^p} dx$$
    $$\int_1^{A} \frac{1}{x^p} dx=\begin{cases}
            \frac{A^{1-p}-1^{1-p}}{1-p}, & p\not=1 \\
            \ln A - \ln 1,               & p=1
        \end{cases}$$
    $$\lim_{A\to\infty}\int_1^{A} \frac{1}{x^p} dx=\begin{cases}
            \frac{1}{p-1}, & p>1 \\
            +\infty,       & p<1 \\
            +\infty,       & p=1
        \end{cases}$$

    $p>1$ --- интеграл сходится, $p\leq 1$ --- интеграл расходится.
\end{example}


\subsection{Критерий Больцано--Коши сходимости несобственного интеграла}
https://youtu.be/OKlsGDA9VYA?t=5884

Пусть $-\infty<a<b\leq+\infty$, f - допустима на $[a, b)$
Тогда сходимость интеграла $\int_a^{\rightarrow b}$
равносильна условию:

$\forall \epsilon > 0 \quad \exists \delta \in (a,b) \quad \forall A, B \in (\delta, b) \quad |\int_A^B f| < \epsilon$
\begin{proof}
    Тривиально из определения предела.
\end{proof}

\newpage
\subsection{Гамма функция Эйлера.}
https://youtu.be/v-Kjgwc6uSU?t=168

$\Gamma$ --- \textbf{гамма-функция Эйлера}
$$\Gamma(t)=\int_0^{+\infty} x^{t-1}e^{-x}dx$$

\subsubsection*{Свойства}
    \begin{enumerate}
        \item При $t > 0$ интеграл сходится
        \item $\Gamma(x + 1) = x \Gamma(x), \Gamma(1) = 1 \implies \Gamma(n + 1) = n!$
        \item $\Gamma(t) - $ выпуклая по t.
        \quad $$f(\alpha t_1 + (1-\alpha)t_2)\leq \alpha f_x(t_1) + (1-\alpha)f_x(t_2)$$
        $$\int_0^{+\infty} x^{(\alpha t_1 + (1-\alpha)t_2)-1}e^{-x}dx\leq \alpha \int_0^{+\infty}x^{t_1-1}e^{-x}dx+(1-\alpha)\int_0^{+\infty}x^{t_2-1}e^{-x}dx$$
        $\implies \Gamma(t)$ - непрерывна
        \item 
        $\Gamma(t) = \frac{\Gamma(t+1)}{t} \stackrel{t\to +0}{=}\frac{1}{t}$

    \end{enumerate}
\newpage

\end{document}