\documentclass[../main.tex]{subfiles}
\graphicspath{{\subfix{../Images/}}}
\begin{document}


\subsection{Верхний и нижний пределы}
https://goo.su/62FQ

\begin{itemize}
    \item Дана последовательность $x_n$.
    \item $y_n:=\sup(x_n, x_{n+1}, x_{n+2},\ldots)$
    \item $z_n:=\inf(x_n, x_{n+1}, x_{n+2},\ldots)$
\end{itemize}

\begin{itemize}
    \item \textbf{Верхний предел} $x_n$: $\overline{\lim\limits_{n\to+\infty}} x_n = \lim\limits_{n\to+\infty}y_n$
    \item \textbf{Нижний предел} $x_n$: $\underline{\lim\limits_{n\to+\infty}} x_n = \lim\limits_{n\to+\infty}z_n$
\end{itemize}


\subsection{Частичный предел}
    Если существует попоследовательность сходящаяся к точке, то эта точка есть частичный предел.
    \newline
    
    Частичный предел вещественной последовательности $x_n$ --- предел вдоль подпоследовательности $n_k$:
    $$n_k\to+\infty, n_1<n_2<\ldots \quad \lim x_{n_k}\in\overline\R$$


\subsection{Абсолютно сходящийся интеграл, ряд}
$f$ --- допустимая функция на $[a, b)$
\newline
\newline

Интеграл $\int_a^b (f)$ -- абсолютно сходится, если:
\begin{enumerate}
    \item $\int_a^b (f)$ сходится
    \item $\int_a^b (|f|)$ сходится
\end{enumerate}

Ряд $A$ абсолютно сходится, если:
\begin{enumerate}
    \item $\sum a_n$ сходится
    \item $\sum |a_n|$ сходится
\end{enumerate}


\subsection{Числовой ряд, сумма ряда, сходимость, расходимость}
Числовой ряд:
\newline

$a_1+a_2+a_3\ldots $, $\sum\limits_{i=1}^{+\infty} a_i$ -- числовой ряд ($a_i\in\R$)
\newline

Частичная сумма:
\newline

$\forall N\in\N \quad S_n:=\sum\limits_{i=1}^n a_i$ -- частичная сумма
\newline

Сходимость ряда:
\newline

Если $\exists \lim_{N\to+\infty} S_n=S\in\R$, ряд сходится, иначе ряд расходится.


\subsection{$n$-й остаток ряда}
Ряд с $N$-ого члена
\newline

$\sum\limits_{k=N}^{+\infty} a_k$ -- $N$-й остаток ряда


\end{document}