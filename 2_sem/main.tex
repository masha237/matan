\documentclass[12pt, a4paper]{article}
%\providecommand{\pgfsyspdfmark}[3]{}
\usepackage[T2A]{fontenc}
\usepackage[utf8]{inputenc}

\usepackage{fancyhdr} 
\usepackage{graphicx}
\usepackage{amsmath}
\usepackage{amsthm}
\usepackage{amsfonts}
\usepackage{amssymb}
\usepackage{xcolor}
\usepackage{bbm}
\usepackage{ulem}
\usepackage{wasysym}
\usepackage{bbold}
\usepackage{dsfont}
\usepackage{amsfonts}
\usepackage{wrapfig}
\usepackage[english,russian]{babel} 
\usepackage{chngcntr}
\usepackage[left=2.3cm, right=2.3cm, top=2.7cm, bottom=2.7cm, bindingoffset=0cm, headheight=15pt]{geometry} 
\usepackage{subfiles}

% \usepackage{background}
% \usepackage{keystroke}

\renewcommand{\footrulewidth}{0.4pt}

\newcommand{\bnd}{\\[1ex]}
\newcommand{\Bnd}{\\[2ex]}
\newcommand{\RomanNum}[1]{\MakeUppercase{\romannumeral #1}}
\newcommand{\eqdef}{\stackrel{\mathrm{def}}{=}}
\newcommand{\scalar}[1]{\langle #1 \rangle}
\newcommand{\abs}[1]{\left| #1 \right|}
\newcommand{\RN}[1]{\sout{roman} #1}

\DeclareMathOperator*{\xor}{\oplus}
\DeclareMathOperator*{\bigxor}{\bigoplus}
% \DeclareMathOperator*{\R}{\mathbb{R}}
\newcommand{\R}{\mathbb{R}}
\DeclareMathOperator*{\Q}{\mathbb{Q}}
\DeclareMathOperator*{\Z}{\mathbb{Z}}
\DeclareMathOperator*{\B}{\mathbb{B}}
\DeclareMathOperator*{\N}{\mathbb{N}}
\DeclareMathOperator*{\0}{\mathbb{0}}
\DeclareMathOperator*{\1}{\mathbb{1}}
\renewcommand{\Vec}{\overrightarrow}
\renewcommand{\U}{\mathbb{U}}
\renewcommand{\C}{\mathbb{C}}

\DeclareMathOperator*{\grad}{grad}
\DeclareMathOperator*{\Interior}{Int}

\theoremstyle{plain}
\newtheorem{theorem}{Теорема}
\newtheorem*{theorem*}{Теорема}
\newtheorem{axiom}{Аксиома}
\newtheorem{lemma}{Лемма}
\counterwithin*{theorem}{subsection}
\counterwithin*{axiom}{subsection}
\counterwithin*{lemma}{subsection}

\theoremstyle{definition}
\newtheorem*{definition}{Определение}

\theoremstyle{remark}
\newtheorem*{remark}{Примечание}
\newtheorem*{exercise}{Упражнение}
\newtheorem*{consequence}{Следствие}
\newtheorem*{example}{Пример}

\newcommand{\candle}{\ensuremath{\accentset{\scalebox{.5}{\(\varspadesuit\)}}{\scalebox{.6}{\(\talloblong\)}}}}
\documentclass[12pt, a4paper]{article}
\usepackage[T2A]{fontenc}
\usepackage[utf8]{inputenc}

\usepackage{fancyhdr} 
\usepackage{lastpage}
\usepackage{graphicx}
\usepackage{amsmath}
\usepackage{amsthm}
\usepackage{amsfonts}
\usepackage{amssymb}
\usepackage{xcolor}
\usepackage{bbm}
\usepackage{ulem}
\usepackage{wasysym}
\usepackage{bbold}
\usepackage{dsfont}
\usepackage{amsfonts}
\usepackage{wrapfig}
\usepackage[english,russian]{babel} 
\usepackage{chngcntr}
\usepackage[left=2.3cm, right=2.3cm, top=2.7cm, bottom=2.7cm, bindingoffset=0cm, headheight=15pt]{geometry} 
\usepackage[colorlinks=true,linkcolor=blue]{hyperref}
\usepackage{draftwatermark}
\SetWatermarkText{В матане как на войне}
\SetWatermarkScale{2}
\SetWatermarkColor[gray]{0.9}
% \usepackage{lmodern}

\usepackage{background}
\usepackage{etoolbox}
\usepackage{keystroke}

\makeatletter
\patchcmd{\tableofcontents}{\@starttoc{toc}}{\hypertarget{totoc}{}\@starttoc{toc}}{}{}
\makeatother

\backgroundsetup{
scale=1,
angle=0,
color=black,
position=current page.south,
vshift=20pt,
contents={
  \tikz[remember picture,overlay]
    \node[inner sep=0pt] {\hyperlink{totoc}{\Return}};
  }
}

\renewcommand{\footrulewidth}{0.4pt}
% \setlength{\parindent}{0em}
% \setlength{\parskip}{1em}

\pagestyle{fancy}
\rfoot{\thepage\ из \pageref{LastPage}}
\cfoot{}

\newcommand{\bnd}{\\[1ex]}
\newcommand{\Bnd}{\\[2ex]}
\newcommand{\RomanNum}[1]{\MakeUppercase{\romannumeral #1}}
\newcommand{\eqdef}{\stackrel{\mathrm{def}}{=}}
\newcommand{\scalar}[1]{\langle #1 \rangle}
\newcommand{\abs}[1]{\left| #1 \right|}

\DeclareMathOperator*{\xor}{\oplus}
\DeclareMathOperator*{\bigxor}{\bigoplus}
\DeclareMathOperator*{\R}{\mathbb{R}}
\DeclareMathOperator*{\Q}{\mathbb{Q}}
\DeclareMathOperator*{\Z}{\mathbb{Z}}
\DeclareMathOperator*{\B}{\mathbb{B}}
\DeclareMathOperator*{\N}{\mathbb{N}}
\DeclareMathOperator*{\0}{\mathbb{0}}
\DeclareMathOperator*{\1}{\mathbb{1}}
\renewcommand{\Vec}{\overrightarrow}
\renewcommand{\U}{\mathbb{U}}
\renewcommand{\C}{\mathbb{C}}

\theoremstyle{plain}
\newtheorem{theorem}{Теорема}
\newtheorem*{theorem*}{Теорема}
\newtheorem{axiom}{Аксиома}
\newtheorem{lemma}{Лемма}
\counterwithin*{theorem}{subsection}
\counterwithin*{axiom}{subsection}
\counterwithin*{lemma}{subsection}

\theoremstyle{definition}
\newtheorem*{definition}{Определение}

\theoremstyle{remark}
\newtheorem*{remark}{Примечание}
\newtheorem*{exercise}{Упражнение}
\newtheorem*{consequence}{Следствие}
\newtheorem*{example}{Пример}

\newcommand{\candle}{\ensuremath{\accentset{\scalebox{.5}{\(\varspadesuit\)}}{\scalebox{.6}{\(\talloblong\)}}}}

\title{\textbf{В матане как на войне} \\ Учебное пособие о том как затащить у Кохася К. П.\\Том~\RN{2}}
\lhead{}

% Впиши сюда себя. Страна должна знать своих героев!
\author{ 
    @irdkwmnsb - Альжанов Максим   \\%
    @dalvikk - Владислав Ковальчук \\%
    @gleb\_shnshn - Шаньшин Глеб \\%
    @RahimHakimov - Хакимов Рахим \\%
    @JelluSandro - Шемякин Никита \\%
    @erove - Еров Егор \\%
    @NULL3301 - Нагибин Вадим \\%
    @GrigorenkoPV - Григоренко Павел \\%
    @maksim shekhunov - Шехунов Максим \\%
    @artyom448 - Фитисов Артём \\%
    Copyright \textcopyright \space \href{https://github.com/gaporf}{Nikolai Akimov}
}

\date{May 2021}

\begin{document}
\maketitle

\newpage
\tableofcontents

\newpage
\section*{Как это редактировать?}
Распределение вариантов \\
TODO \\

\noindent Картинки можно вставлять так: \\
% scale --- масштабирование. Подробнее в интернете
\includegraphics[scale = 0.5]{Images/smile.jpeg}

\noindent Если что то похерилось --- писать Максиму (@irdkwmnsb) \\

\noindent $a \ b$  --- маленький пробел в режиме набора мат формул \\
$a \quad b$   --- средний \\ 
$a \qquad b$  --- большой \\

Лайфхак: дважды нажмите на текст в пдфке и ваш курсор переместится на ее латех код

Комент

\noindent Как написать комент: В правом верхнем углу есть кнопка Review, после ее нажатия откроется поле где видны комментарии + выделив текст можно можно добавить новый тыкнув на Add comment
\newpage
\section{Формулировки и определения}
% Так как нумерация строк в табличке Кохася с 3
% Добавь \lhead{ФАМИЛИЯ} чтобы забить билет
\setcounter{subsection}{2}
\lhead{Григоренко}
\subfile{Terms/03-07} % готово

\lhead{Утюжников}
\subfile{Terms/08-12} % готово

\lhead{Шехунов} % 17 Фитисов, \lhead внутри
\subfile{Terms/13-17} % готово

\lhead{Григоренко}
\subfile{Terms/18-22} % готово

\lhead{Нагибин}
\subfile{Terms/23-27} % готово

\lhead{Акулов}
\subfile{Terms/28-32} % готово

\lhead{Григоренко}
\subfile{Terms/33-37} % готово

\lhead{Григоренко}
\subfile{Terms/38-42} % готово

\lhead{Григоренко}
\subfile{Terms/43-47} % готово

\lhead{Григоренко}
\subfile{Terms/48-52} % готово

\lhead{Леонов}
\subfile{Terms/53-54} % готово

\newpage
\section{Теоремы}
\setcounter{subsection}{2}
\lhead{Шаньшин} 
\subfile{Theorems/03-07} % готово

\lhead{Утюжников}
\subfile{Theorems/08-12} % готово

\lhead{Нагибин}
\subfile{Theorems/13-17} % готово

% 18 Волков, 19-22 кто? 
\subfile{Theorems/18-22} % готово
\lhead{M3137-y2019}
\subfile{Theorems/23-27} % готово

\lhead{Акулов}
\subfile{Theorems/28-32} % готово

% 33-34 Еров, 35-37 Григоренко. \lhead внутри 
\subfile{Theorems/33-37} % готово

\lhead{Шаньшин} 
\subfile{Theorems/38-42} % готово

\lhead{Шаньшин}
\subfile{Theorems/43-47} % готово

\lhead{Шаньшин}
\subfile{Theorems/48-52} % готово

\lhead{Жогова}
\subfile{Theorems/53-57} % готово

\lhead{Жогова}
\subfile{Theorems/58-62} % готово

\lhead{Жогова} % 67 Григоренко \lhead внутри
\subfile{Theorems/63-67} % готово

\lhead{Шаньшин}
\subfile{Theorems/68-72} % готово без 72. Метод Лапласа

\lhead{А кто?}
\subfile{Theorems/73-77} % готово

\lhead{Леонов}
\subfile{Theorems/78-79} % готово

\end{document}
 