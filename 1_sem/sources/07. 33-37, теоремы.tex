\newpage
\rhead{Нагибин}
\subsection{Критерий Больцано-Коши для последовательностей и отображений}
\begin{definition}
$X$ {---} м. п., $X$ {---} полное $\Leftrightarrow \forall \{x_n\}$ фундаментальная сходится в $X$ 
\end{definition}
\begin{theorem*}
Пусть $f: D \subset X \rightarrow Y$, $Y$ {---} полное, $a$ {---} предельная точка $D$:
$$\exists \lim_{x \to a}{f(x)} \Leftrightarrow \forall \varepsilon > 0 \ \exists V(a) \ \forall x_1, x_2 \in V(a) \ \rho(x_1, x_2) < \varepsilon$$
\end{theorem*}
\begin{proof}
$(\Rightarrow)$ Доказывается аналогично переходу от предела к сходимости к себе (по факту, это абсолютно тоже самое), т. е. достаточно записать определение и проявить волю к победе!
  
$(\Leftarrow)$ По Гейне.
\end{proof}
\begin{remark}
Теорема записана для отображений в терминах окрестностей, для последовательностей утверждение тривиальнее в силу полноты образа отображения.
\end{remark}

\newpage
\subsection{Теорема о пределе монотонной функции}
\begin{theorem}
Пусть $f: D \subset \R \to \R$ и монотонная, $a$ {---} предельная точка, $D_1 = D \cup (-\infty; a)$ тогда:
\begin{enumerate}
    \item $f$ {---} возрастает и ограничено сверху на $D_1 \Rightarrow \exists$ предел {---} $f(x - a)$
    \item $f$ {---} убывает и ограничено снизу на $D_1 \Rightarrow \exists$ предел {---} $f(x - a)$
\end{enumerate}
\end{theorem}
\begin{proof}
Положим $A := \underset{x \in D_1}{sup}f(x)$. Осталось проверить $f(x) \to A \in \R$.
  
По техническому определению супремума $\forall \varepsilon > 0 \ \exists x_1 \in D_1: A - \varepsilon < f(x_1) \leq A$. Берём $\delta = a - x_1 \Rightarrow \forall x \in D_1: a - \delta < x < a \Rightarrow x_1 < x < a$ отсюда в силу монотонности $A - \varepsilon < f(x_1) \leq f(x) \leq A$
\end{proof}

\newpage
\subsection{Теорема о замене на эквивалентную при вычислении пределов. Таблица эквивалентных}
\begin{theorem*}
Пусть $X$ {---} м. п, $f, \widehat{f}, g, \widehat{g}: D \subset X \rightarrow \R (\C)$, $x_0$ {---} предельная точка $D$, $f(x) \sim \widehat{f}(x)$, $g(x) \sim \widehat{g}(x)$ при $x \to x_0$. Тогда справедливы следующие утверждения:
\begin{enumerate}
    \item \[ \lim_{x\to x_0}{f(x)g(x)} = \lim_{x \to x_0}{\widehat{f}(x)\widehat{g}(x)} \]
    \item Если $x_0$ {---} предельная точка области определения $\displaystyle \frac{f}{g}$, $g(x) \neq 0$ и $\widehat{g}(x) \neq 0$, то
    \[ \lim_{x\to x_0}{\frac{f(x)}{g(x)}} = \lim_{x\to x_0}{\frac{\widehat{f}(x)}{\widehat{g}(x)}} \]
\end{enumerate}
\end{theorem*}
\begin{proof}Доказывается, раскладывая обе функции по определению эквивалентных функци.
\end{proof}

\newpage
\subsection{Теорема единственности асимптотического разложения}
\begin{theorem*}
Пусть $f, g_k: D \subset X \rightarrow \R$, $x_0$ {---} предельная точка $D$, $\forall k \ g_{k + 1} = o(g_k)$, $\exists U(x_0) \ \forall k \ g_k \neq 0$ для $x \in \dot U(x_0) \cap D$. Если
$$f(x) = \sum_{k = 0}^{n}c_kg_k + o(g_n)$$
$$f(x) = \sum_{k = 0}^{n}d_kg_k + o(g_n)$$
  
Тогда $\forall k \ c_k = d_k$.
\end{theorem*}
\begin{proof}
Возьмём первый индекс, на котором соответсвующие коэффиценты различаются и обозначим его за $m$, далее вычтем одно выражение из другого:
$$0 = (c_m - d_m)g_m + \sum_{k = m + 1}^{n}{(c_k - d_k)g_k} + o(g_n) = (c_m - d_m)g_m + o(g_m) \Rightarrow 0 = (c_m - d_m) + \frac{o(g_m)}{g_m}$$
  
Отсюда, используя предельный переход, получаем, что такого индекса не существует, а значит все коэффиценты совпадают.
\end{proof}

\newpage
\subsection{Арифметические свойства непрерывных отображений, теорема о стабилизация знака}

\begin{theorem} (Арифметические свойсвта непрерывных отображений)
Пусть $f, g: D \subset X \rightarrow Y$ (нормаированное пространство), $x_0 \in D$, $\lambda: D \rightarrow \R$, $f, g, \lambda$ {---} непрерывны в $x_0$. Тогда $f \pm g, \lambda f, ||f||$ {---} непрерывны в $x_0$.
\end{theorem} 
\begin{remark}
Если к тому же $\displaystyle g(x_0) \neq 0 \Rightarrow \frac{f(x)}{g(x)}$ {---} непрерывна в $x_0$.
\end{remark}
\begin{proof}
Если $x_0$ является изолированной точкой $D$, то всё выполняется, в противном случае пользуемся арифметическими свойсвтами предела и получаем аналогичный результат.
\end{proof}

\begin{theorem} (О стабилизации знака)
Пусть $f: D \subset X \rightarrow \R$, $x_0 \in D$, $f$ {---} непрерывна в $x_0$ и $f(x_0) \neq 0$. Тогда $\exists U(x_0): \forall x \in U(x_0) \ \operatorname{sign}(x) = \operatorname{sign}(x_0)$ 
\end{theorem}
\begin{proof}
Тривиально по определению предела.
\end{proof}