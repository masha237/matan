 \newpage
\rhead{Ковальчук}
\subsection{Теорема о компактности в пространстве и в подпространстве}
 Пусть $(X,\rho)$ --- метрическое пространство, $Y\subset X$ --- подпространство, $K\subset Y$ \\
	    Тогда $K$ --- компактно в $Y \Leftrightarrow K$ --- компактно в $X$.
	
	\begin{proof} \nobreakspace
	\begin{itemize}
	    \item $\Rightarrow$
		
		Пусть $K \subset \bigcup\limits_{\alpha \in A} G_\alpha, $ где $G_\alpha$ --- открытые в $X$ \\
		Тогда так как $K \subset Y: K \subset \bigcup\limits_{\alpha \in A} \left(G_\alpha \cap Y\right) \Rightarrow \exists \alpha_1, \ldots \alpha_n: K \subset \bigcup\limits_{i=1}^{n}(G_{\alpha_i} \cap Y)$ (т.е. существует конечное подмножество, так как $K$ компактно в $Y$). И раз $K \subset \bigcup\limits_{i=1}^{n}(G_{\alpha_i} \cap Y)$, то тем более $K \subset \bigcup\limits_{i=1}^{n}G_{\alpha_i}$
		\item $\Leftarrow$
	
	    Дано: $K$ --- компактно в $X$, правда ли что $K$ --- компактно в $Y$?
	    $$K\subset\bigcup\limits_{\alpha\in A} G_\alpha, G_\alpha\text{ --- открытые в }Y$$
	    $$\exists \tilde G_\alpha \textit{ --- открыто в $X$} : G_\alpha=\tilde G_\alpha\cap Y \Rightarrow \text{[по Т. об открытых и замкнутых множествах  ]} \Rightarrow$$ $$ K \subset \bigcup \tilde G_\alpha \Rightarrow \text{[по компактности в $X$]} \Rightarrow $$
	    $$\Rightarrow K \subset \bigcup\limits_{i = 1}^{n} \tilde G_\alpha \Rightarrow  K \subset \bigcup\limits_{i = 1}^{n}  G_\alpha$$
	   \end{itemize}
	\end{proof}
	\subsection{Простейшие свойства компактных множеств}
Пусть $X$---метрическое пространство, $K \subset X$. Тогда:
		\begin{enumerate}
			\item $K$--- компактно $\Rightarrow$ замкнуто + ограничено
			\item $Y \subset X, Y$---компактно, $K$ замкнуто в $Y \Rightarrow K$---компактно (замкнутое подмножество компактного множества компактно)
		\end{enumerate}
	
	\begin{proof} \nobreakspace
		\begin{enumerate}
		\item	
			\begin{enumerate}
			\item  Замкнуто ли $K$? Для этого достаточно проверить что $K^c = X \setminus K$ --- открыто \\
			Пусть $a \in K^c$. Окружим каждую точку $K$ каким нибудь шаром, не задевая $a$. Тогда $K \in \bigcup\limits_{x \in K}B\left(x, \frac{1}{2}\rho(x,a)\right)$ --- открытое покрытие $\Rightarrow$ [по компактности] $\exists x_1, \ldots x_n : K = \bigcup\limits_{i=1}^{n}B(x_i, \frac{1}{2} \rho(x_i, a))$ \\
			Возьмем $R:=\ \min\{\frac{1}{2}\rho(x_i, a): 1 \le i \le n \} \Rightarrow$ очевидно, что $B(a, R) \subset K^c$. Таким образом каждая точка $K^c$ входит вместе с некоторым шаром $\Rightarrow K^c$ --- открыто \\
			\item  Ограничено ли $K?$ \ Пусть $a \in X$--- любая точка, $K \subset \bigcup\limits_{n=1}^{+\infty} B(a,n)$ [формально это открытое подпокрытие, тогда по компактности] $\Rightarrow \exists n_1 \ldots n_l: K \subset \bigcup\limits_{n=1}^{l}B(a, n_l)$ \\
			Ну и тут написано что $a$ содержится в каком то шаре большого радиуса, если взять наибольший
		\end{enumerate}
		\item Проверим, компактно ли $K$ в $Y$\\
		$K \subset \bigcup\limits_{\alpha \in A} G_\alpha$ --- открытые в $Y$ \\
		$K$ --- замкнуто, значит $K^c$ открыто $\Rightarrow Y \subset $ (на самом деле =) $K^c \cup \bigcup\limits_{\alpha \in A} G_\alpha$ (это открытое покрытие) $\Rightarrow \exists$ конечное открытое подпокрытие $Y$: \\
		$Y \subset \bigcup\limits_{i=1}^n G_{\alpha_i} $, и, возможно, $\cup  K^c$. Тогда \\
		$K \subset \bigcup\limits_{i=1}^n G_{\alpha_i} $
		\end{enumerate}
	\end{proof}
	
	\begin{remark}
	В $X = \R^m K$ --- компактно $\Leftrightarrow $ замкнуто + ограниченно, но в любом $X$ ( и даже подпространстве $\R^m)$ это неверно! 	
		\begin{example}
		$X = (0,1)$ --- ограничено и замкнуто (в $X$), но некомпактно, так как можем взять следующее открытое покрытие: 
		$$X = \bigcup_{k = 1}^{\infty} \left(\frac{1}{k+2}; \frac{1}{k}\right)$$	
		\end{example}
	\end{remark}
	
	\begin{remark}
		$K \subset X, K$ --- конечное множество, тогда очевидно, что $K$ --- компактно	
	\end{remark}

\subsection{Лемма о вложенных параллелепипедах}
Параллелепипед: $[a,b] = \{x \in \R^m: \forall i: a_i \le x_i \le b_i \}$
		\\
		$[a_1, b_1] \supset [a_2, b_2] \supset [a_3, b_3]  \ldots $ \\
		Тогда $\bigcap\limits_{i = 1}^{\infty} [a_i, b_i] $ --- непусто
	
	\begin{proof}
		Рассмотрим покоординатно: \\
		$\forall i = 1 \ldots m, [(a_k)_i, (b_k)_i] \supset [(a_{k+1})_i, (b_{k+1})_i] \ldots $ ---к этой системе вложенных промежутков применим аксиому Кантора. \\
		$\exists c_i \subset \bigcap\limits_{i=1}^{\infty} [(a_k)_i, (b_k)_i]$ \\
		Тогда, очевидно, $c = (c_1 \ldots c_m) \subset \bigcap\limits_{i = 1}^{\infty} [a_i, b_i]$, так как $\forall i : (a_k)_i \le c_i \le (b_k)_i$
	\end{proof}

\subsection{Компактность замкнутого параллелепипеда в $\R^m$}
Пусть $K = [a,b]$ --- замкнутый параллелепипед в $\R^m. $ Тогда $K$ --- компактно
	  
	  \begin{proof}
	  	$[(a,b)] \subset \bigcup G_\alpha $ --- открытое покрытие в $\R^m$ \\
	  	Допустим из этого открытого покрытия невозможно выбрать конечное подпокрытие. Тогда осуществляем половинное деление. Тогда обязательно (так как мы предположили что конечного покрытия не существует) будет существовать четвертинка (в двумерном случае, в произвольном $\frac{1}{2^m}$ часть) которую нельзя накрыть конечным покрытием множеств. Запускаем такой алгоритм, изначально $a_1 = a, b_1 = b$  \\
	  	Получилась цепочка вложенных параллелепипедах. Тогда лемма о вложенных параллелепипедах: \\
	  	$\exists c \in \bigcap[a_k, b_k]$. \\
	  	$diam [a_k, b_k] = \frac{diam [a_1, b_1]}{2^{k-1}}$, очевидно, длина диаметра стремится к нулю \\
	  Тогда $\exists G_\alpha : c \in G_\alpha$, и так как $G_\alpha$ открытое, то $c$ входит с некоторой окрестностью $\Rightarrow \exists B(c,r) \subset G_\alpha$, и когда диаметр параллелепипеда станет меньше $r$, мы получим что весь параллелепипед вместе с точкой $c$ содержится в этом одном шаре $G_\alpha$, но мы ведь строили параллелепипеды так, чтобы их нельзя было накрыть конечным подпокрытием. Мы получили противоречие.  
	  \end{proof}
	  
\subsection{Теорема о характеристике компактов в $\R^m$}
Дано множество $K$ лежащее в $R^m: K \subset R^m$. Тогда следующие утверждения эквивалетны:
	\begin{enumerate}
	  \item $K$ --- замкнуто и ограничено (Мы доказывали что компактное множество обязательно замкнуто и ограничено, но в $R^m$ это работает еще в обратную сторону)
	  \item $K$ --- компактно
	  \item $K$ --- секвенциально компактно. Это значит, что $\forall (x_n) \in K  \ \exists n_k$ --- строго возрастающая последовательность номеров, $\exists x \in K: x_{n_k} \to x$ (у любой последовательности имеется сходящаяся подпоследовательность, причем $x$ тоже должен лежать в $K$)
	\end{enumerate}
	
	\begin{proof} \nobreakspace \\
		\begin{itemize}
		\item 1 $\Rightarrow 2$ \\
	  		 $K$ --- замкнуто и (ограничено, значит можем заключить в шаре, да и не только в шаре, давайте в параллелепипеде) содержится в параллелепипеде. Замкнутое подмножество компактного множества компактно (по теореме о простейших свойствах) $\Rightarrow K$--- компактно \\
	  	\item 2 $\Rightarrow 3$ \\
			Пусть дана последовательность $x_n$. Можем ли мы найти такую подпоследовательность $x_{n_k}$? Разберем два случая:
			\begin{enumerate}
				\item Множество значений $x_n$ конечно. Очевидно, можем. Допустим у нас есть 10 значений $x_n$, а самих номеров бесконечно много. Значит одному из значений отвечает бесконечно много номеров. Берем эти номера, это и будут $n_k$, значит $\exists$ бесконечная стационарная подпоследовательность. 
				\item Множество значений $x_n$ бесконечно. $D = \{ x_n \} $. \\ 
				Предположим что  у $D$ нет предельных точек в $K$, тогда построим покрытие: \\
				$K = \bigcup\limits_{x \in K} B(x, \varepsilon_x)$ \\
				$x \in K$ не предельная точка для $D \Rightarrow$
				 можем окружить таким шаром,  что там нет точек $D: \exists \varepsilon_x : \dot B(x, \varepsilon_x) $ не пересекается с $D$. \\
				Тогда это открытое покрытие. У этого открытого покрытия нет конечного подпокрытия,  потому что множество $D$ бесконечно и не может быть покрыто конечным количеством шаров. Но это противоречие, так как $K$ компактно и значит у любого открытого покрытия есть конечно подпокрытие $\Rightarrow$ у $D$ есть предельные точки
				
				Тогда пусть $x$ --- предельная точка. Значит $\exists x_{m_k} \to x, $ где $x_{m_k} \in D, x_{m_k} \neq x$ (ко всякой предельной точке можно подойти не наступая на саму точку) \\
				Последовательность номеров должна быть возрастающей, поэтому отсортируем последовательность $m_k$ и удалим повторы
		
				Рассмотрим $x_{m_1}: \exists K \ \forall k > K : \abs{x_{m_k} - x} < \underbrace{\abs{x_{m_1} - x}}_{\varepsilon}$ \\
				То есть при $k > K \ m_k \neq m_1 \Rightarrow m_1$ встретится конечное число раз
				Аналогично  $\forall i: m_i$ встречается в последовательности ($m_k$) конечное число раз \\
				Итого, алгоритм построения $n_k$:
				Берем $m_1$, при $k > K_1$ выбираем наименьшее значение $m_l$ \\
				Обозначим $n_1 = 1, n_2 = l$. Аналогично запускаем выше проделанное наблюдение \\
				$\exists K_1$ при $k > K_1$ : $m_i \neq m_l$, берем наименьшее значение $m_i$, обозначим $n_2$ 
				\ldots
			\end{enumerate}
			
			\item $3 \Rightarrow 1$ \\
			Может ли $K$ быть неограниченно? \\
			Тогда $\forall n \ \exists x_n \in K: \|x_n\| > n $. Возьмем шар радиусом $n$ в нуле, $K$ из него вылезает. Возьмем $x_n$ за пределами шара и так при каждом $n$.  Получили какую то последовательность. Но тут нет сходящейся подпоследовательности, потому что если $x_n \to x$, то $\|x_{n_k}\| \to \|x\|$, а у нас $\|x_{n_k}\| \to + \infty  \Rightarrow K$ --- ограниченно \\
			Замкнуто ли $K$? А что если $\exists a$ --- предельная точка $K, a \not \in K$ \\
			К предельной точке всегда можно подойти сколько угодно близко $\Rightarrow \exists x_n \to a, x_n \in K$. Любая подпоследовательность должна стремится к $a$, но согласно секвенциальной компактности, $a \in K$, но у нас $a \not \in K \Rightarrow$ противоречие. Значит все предельные точки лежат в $K$. 
		\end{itemize}
	\end{proof}
	
	\begin{remark}
		Утверждение 2 равносильно утверждению 3 в любом метрическом пространстве. Из 2 следует 1 по теореме о простейших свойствах компактов. Но из 1, к сожалению,  не следует 2: \\
		Рассмотрим интервал (0,1). Он не компактен,  ограничен и замкнут если мы его рассматриваем в себе (как самостоятельное пространство). 
	\end{remark}
