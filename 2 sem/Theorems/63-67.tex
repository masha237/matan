\documentclass[../main.tex]{subfiles}
\graphicspath{{\subfix{../Images/}}}
\begin{document}


\subsection{Лемма об оценке приближения экспоненты ее замечательным пределом}
\subsection*{Лемма 1}
        
            \subsubsection*{Формулировка}
            
                $\Pi(n, x) = \int\limits^n_0 \left(1 - \dfrac{t}{n} \right)^n t^{x - 1} dt$, где $x > 0$
                
                Тогда $\Pi(n, x) = \dfrac{1 \cdot 2 \cdot \ldots \cdot n}{x(x + 1)\ldots(x+n)} \cdot n^x$
                
            \subsubsection*{Доказательство}
            
                $\Pi(n, x) = n^x \int\limits^1_0 (1 - s)^n \cdot s^{x - 1} ds = n^x \left(  (1 - s)^n \cdot \dfrac{s^x}{x} \bigg|^{s = 1}_{s = 0} + \dfrac{n}{x} \int\limits^1_0 (1 - s)^{n - 1} \cdot s^x ds \right) = n^x \cdot \dfrac{n}{x} \int\limits^1_0 (1 - s)^{n - 1} s^x ds = n^x \cdot \dfrac{n}{x} \cdot \dfrac{n - 1}{x + 1} \cdot \int\limits^1_0 (1 - s)^{n - 2} s^{x - 1} ds = \ldots$ получаем то, что хотели
                
        \subsection*{Лемма 2}
        
            \subsubsection*{Формулировка}
            
                При $0 \leq t \leq n$ 
                
                $0 \leq e^{-t} - \left(1 - \dfrac{t}{n} \right)^n \leq \dfrac{1}{n} t^2 e^{-t}$
                
            \subsubsection*{Доказательство}
            
                $(1 + y) \leq e^y \leq (1 - y)^{-1}$, $y \in [0, 1]$ в силу выпуклости $e^x$
                
                $e^y \geq 1 + y$
                
                $e^{-y} \geq 1 - y$
                
                возведём в $(-n)$, $y := \dfrac{t}{n}$
                
                $\left( 1 + \dfrac{t}{n} \right)^{-n} \geq e^{-t} \geq \left(1 - \dfrac{t}{n} \right)^n$
                
                $0 \leq e^{-t} - \left( 1 - \dfrac{t}{n} \right)^n = e^{-t} \left(1 - e^t \left(1 - \dfrac{t}{n} \right)^n \right) \leq e^{-t} \left(1 - \left(1 + \dfrac{t}{n} \right)^n \left( 1 - \dfrac{t}{n}\right)^n\right)$
                
                $e^{-t} \left(1 - \left(1 - \dfrac{t^2}{n^2} \right)^n \right) \leq \dfrac{t^2}{n} e^{-t}$ (это неравенство Бернулли)
          
\newpage


\subsection{Формула Эйлера для гамма-функции}

        \subsection*{Формулировка}
        
            При $x > 0$ верно, что
            
            $$\lim\limits_{n \rightarrow +\infty} \dfrac{1 \cdot 2 \cdot \ldots \cdot n}{x(x+1)\ldots(x+n)} \cdot n^x = \Gamma(x)$$
            
        \subsection*{Доказательство}
        
            $\Gamma(x) - \lim\limits_{n \rightarrow +\infty} \Pi(n, x) = \lim\limits_{n \rightarrow +\infty} \left( \int\limits^n_0 \left(e^{-t} - \left(1 - \dfrac{t}{n} \right)^n \right) t^{x - 1} dt + \int\limits^{+\infty}_n t^{x - 1} e^{-t} dt \right)$
            
            $\int\limits^{+\infty}_n e^{-t} t^{x - 1} dt \xrightarrow[n \rightarrow +\infty]{} 0$
            
            $\int\limits^n_0 \dfrac{1}{n} e^{-t} t^2 t^{x - 1} dt \leq \dfrac{1}{n} \int\limits^{+\infty}_0 t^{x + 1}e^{-t}dt \rightarrow 0$
            
\newpage


\subsection{Формула Вейерштрасса для Г-функции}

        \subsection*{Формулировка}
        
            Пусть $x > 0$, $\gamma$ ~--- постоянная Эйлера. Тогда
            
            $$\dfrac{1}{\Gamma(x)} = x e^{\gamma x} \prod\limits^{+\infty}_{k = 1} \left( 1 + \dfrac{x}{k} \right) e^{-\frac{x}{k}}$$
            
        \subsection*{Доказательство}
        
            $\dfrac{1}{\Gamma(x)} = \lim\limits_{n \rightarrow +\infty} n^{-x} \dfrac{x(x+1)\ldots(x+n)}{1 \cdot 2 \ldots \cdot n} = \lim\limits_{n \rightarrow +\infty} \left( n^{-x} \cdot x \cdot \dfrac{x + 1}{1} \cdot \dfrac{x + 2}{2} \cdot \ldots \cdot \dfrac{x + n}{n} \right) = \lim\limits_{n \rightarrow +\infty} x \cdot n^{-x} \cdot \prod^n_{k = 1} \left( 1 + \frac{x}{k} \right) = \lim\limits_{n \rightarrow +\infty} x e^{x \left(1 + \frac{1}{2} + \ldots + \frac{1}{n} \right)} \cdot e^{-x \ln n} \prod^n_{k = 1} \left( 1 + \dfrac{x}{k} \right) e^{-\frac{x}{k}} = x \cdot e^{\gamma x} \prod^{+\infty}_{k = 1} \left( 1 + \dfrac{x}{k} \right) e^{-\frac{x}{k}}$
            
\newpage


\subsection{Вычисление произведений с рациональными сомножителями}

        Пусть $u_n = A \cdot \dfrac{(n + a_1) (n + a_2) \cdot (n + a_k)}{(n + b_1) (n + b_2) \ldots (n + b_l)}$, где $a_i$ и $b_i \in \mathbb{Q}$. Хотим найти $\prod\limits^{+\infty}_{i = 1} u_i$. Самый интересный случай, это то, что произведение сходится, тогда $u_n \rightarrow 1$, а значит $k = l$ и $A = 1$

        $u_n = \dfrac{ \left(1 + \frac{a_1}{n} \right) \left(1 + \frac{a_2}{n} \right) \ldots \left(1 + \frac{a_k}{n} \right)}{\left(1 + \frac{b_1}{n} \right) \left(1 + \frac{b_2}{n} \right) \ldots \left( 1 + \frac{b_k}{n} \right)}$ и при $n \rightarrow +\infty$
        
        $\prod\limits^{+\infty}_{n = 1} \dfrac{ \left(1 + \frac{a_1}{n} \right) \left(1 + \frac{a_2}{n} \right) \ldots \left(1 + \frac{a_k}{n} \right)}{\left(1 + \frac{b_1}{n} \right) \left(1 + \frac{b_2}{n} \right) \ldots \left( 1 + \frac{b_k}{n} \right)} = \prod\limits^{+\infty}_{n = 1} \dfrac{ \left(1 + \frac{a_1}{n} \right) e^{-\frac{a_1}{n}} \left(1 + \frac{a_2}{n} \right) e^{-\frac{a_2}{n}} \ldots \left(1 + \frac{a_k}{n} \right) e^{-\frac{a_k}{n}}}{\left(1 + \frac{b_1}{n} \right) e^{-\frac{b_1}{n}} \left(1 + \frac{b_2}{n} \right) e^{-\frac{b_2}{n}} \ldots \left( 1 + \frac{b_k}{n} \right) e^{-\frac{b_k}{n}}} = \prod\limits^{+\infty}_{i = 1} \dfrac{\frac{1}{\Gamma (a_1) a_1 e^{\gamma a_1}} \cdot \ldots \cdot \frac{1}{\Gamma (a_k) a_k e^{\gamma a_k}}}{\frac{1}{\Gamma (b_1) b_1 e^{\gamma b_1}} \cdot \ldots \cdot \frac{1}{\Gamma (b_k) b_k e^{\gamma b_k}}} = \prod\limits^{+\infty}_{i = 1} \dfrac{\Gamma (1 + b_1) \ldots \Gamma (1 + b_k)}{\Gamma (1 + a_1) \ldots \Gamma(1 + a_k)}$
        
\newpage

\lhead{Григоренко}
\subsection{Разложение синуса в бесконечное произведение}
https://youtu.be/wWtKfrt\_ZQY?t=0s
\begin{lemma}
\begin{equation*}
    n \in \N \Rightarrow \forall x \in \R \quad \sin x =
    (2 \cdot n + 1) \sin \frac{x}{2n + 1}
    \prod_{k=1}^{n}\left(1 - \frac{\sin^2 \frac{x}{2n + 1}}{\sin^2 \frac{\pi k}{2n + 1}}\right)
\end{equation*}
\begin{proof}
\begin{gather*}
    m \overset{\text{def}}{=}2n+1 \\
    \text{Формула Мувара: } \cos mz + i \sin mz= (\cos z + i \sin z)^m \\
    \operatorname{Im}(\cos mz + i \sin mz) = \operatorname{Im}\left(\left(\cos z + i \sin z\right)^m\right)\\
    \sin mz = \binom{m}{1}\cos^{m-1}z\sin z -  \binom{m}{3}\cos^{m-3}z\sin^3 z + \dots\\
    \sin mz = \sin z \cdot P\left(\sin^2 z\right)\text{, т.к. чётные степени косинуса легко выражаются через синус}\\
    \text{Пусть } z \in\left\{\frac{k\pi}{m}:\ k\in\{1,\dots,n\}\right\} \\
    \sin k\pi = \sin \frac{k\pi}{m} \cdot P\left(\sin^2 \frac{k\pi}{m}\right)\\
    0 = \sin \frac{k\pi}{m} \cdot P\left(\sin^2 \frac{k\pi}{m}\right)\\
    \sin\frac{k\pi}{m}\neq 0 \Rightarrow \left\{\sin^2\frac{k\pi}{m}:\ k\in\{1,\dots,n\}\right\}
    \text{~--- множество корней }P\text{ т.к. }\deg P \le n\\
    P(U)=A\prod_{k=1}^{n}\left(1-\frac{u}{\sin^2\frac{k\pi}{m}}\right)\\
    A=P(0)=\lim_{z \to 0}\frac{\sin mz}{\sin z}=m=2n+1 \\
    \sin \left((2n+1) z\right) = (2n+1) \sin z \prod_{k=1}^{n}\left(1-\frac{\sin^2 z}{\sin^2\frac{k\pi}{2n+1}}\right)\\
    z \overset{\text{def}}{=}\frac{x}{2n+1}\\
    \sin x=(2n+1)\sin\frac{x}{2n+1}\prod_{k=1}^{n}\left(1-\frac{\sin^2 \frac{x}{2n+1}}{\sin^2\frac{k\pi}{2n+1}}\right)
\end{gather*}
\end{proof}
\end{lemma}

\begin{lemma}
\begin{equation*}
    \lim_{n \to +\infty}\left(1-\frac{\sin^2 \frac{x}{2n+1}}{\sin^2\frac{k\pi}{2n+1}}\right)=
    1-\frac{x^2}{\pi^2 k^2}
\end{equation*}
\begin{proof}
\begin{equation*}
    \lim_{n \to +\infty}\left(1-\frac{\sin^2 \frac{x}{2n+1}}{\sin^2\frac{k\pi}{2n+1}}\right)=
    \lim_{n \to +\infty}\left(1-\frac{\left(\frac{x}{2n+1}\right)^2}{\left(\frac{k\pi}{2n+1}\right)^2}\right)=
    1-\frac{x^2}{\pi^2 k^2}
\end{equation*}
\end{proof}
\end{lemma}

\begin{lemma}
\begin{equation*}
    \lim_{n \to +\infty}(2n+1)\sin\frac{x}{2n+1}=x
\end{equation*}
\begin{proof}
Очевидно
\end{proof}
\end{lemma}

\begin{theorem}
\begin{equation*}
    \forall x\in R \quad \sin x = x\cdot\prod_{j=1}^{\infty}\left(1-\frac{x^2}{\pi^2 j^2}\right)
\end{equation*}
\begin{proof}
\begin{gather*}
    \text{Возьмём }n,k\in\N:\ n>k \\
    u_k^n \overset{\text{def}}{=}(2n+1)\sin\frac{x}{2n+1}
    \prod_{j=1}^{k}\left(1-\frac{\sin^2 \frac{x}{2n+1}}{\sin^2\frac{j\pi}{2n+1}}\right)\\
    v_k^n\overset{\text{def}}{=}\prod_{j=k+1}^{n}\left(1-\frac{\sin^2 \frac{x}{2n+1}}{\sin^2\frac{j\pi}{2n+1}}\right)\\
    \text{По первой лемме: }\sin x = u_k^n v_k^n\\
    \text{По второй и третьей: }\lim_{n \to +\infty}u_k^n=x\cdot\prod_{j=1}^{k}\left(1-\frac{x^2}{\pi^2 j^2}\right)
    \overset{\text{def}}{=}u_k\\
    \begin{cases}
        \sin x = u_k^n v_k^n\\
        \exists \lim\limits_{n \to +\infty}\sin x = \sin x\\
        \exists \lim\limits_{n \to +\infty}u_k^n = u_k
    \end{cases}
    \Rightarrow
    \begin{cases}
        \exists  \lim\limits_{n \to +\infty}v_k^n \overset{\text{def}}{=}v_k\\
        \sin x = u_k v_k
    \end{cases}\\
    \begin{cases}
        \sin x = u_k v_k\\
        \exists \lim\limits_{k \to +\infty}\sin x = \sin x\\
        \lim\limits_{k \to +\infty}u_k = x\cdot\prod\limits_{j=1}^{\infty}\left(1-\frac{x^2}{\pi^2 j^2}\right)
        \text{ очев. }\exists
    \end{cases}
    \Rightarrow
    \begin{cases}
        \exists  \lim\limits_{k \to +\infty}v_k \overset{\text{def}}{=}v\\
        \sin x = \left[x\cdot\prod\limits_{j=1}^{\infty}\left(1-\frac{x^2}{\pi^2 j^2}\right)\right] v
    \end{cases}
\end{gather*}
Остаётся доказать, что $v=1$.
\begin{gather*}
    0<\phi<\frac{\pi}{2} \Rightarrow \frac{2}{\pi}\varphi < \sin \varphi < \varphi \\
    1>1-\frac{\sin^2\frac{x}{2n+1}}{\sin^2\frac{j}{2n+1}}>
    1-\frac{\frac{x^2}{(2n+1)^2}}{\frac{4}{\pi^2}\cdot\frac{\pi^2 j^2}{(2n+1)^2}}=
    1-\frac{x^2}{4j^2}\\
    1>v_k^n>\prod_{j=k+1}^{n}\left(1-\frac{x^2}{4j^2}\right)\\
    1>v_k\ge\prod_{j=k+1}^{+\infty}\left(1-\frac{x^2}{4j^2}\right)\\
    1\ge v \ge 1 \text{ (остаток сходящегося произведения)}
\end{gather*}
\end{proof}
\end{theorem}
\newpage

\end{document}