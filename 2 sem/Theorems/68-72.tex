\documentclass[../main.tex]{subfiles}
\graphicspath{{\subfix{../Images/}}}
\begin{document}


\subsection{Единственность производной}
Производный оператор единственный\\\\
\textbf{Доказательство:}\\\\
Проверим, что $\forall n \in \mathbb{R}^m$ $L(n)$ задан однозначно\\\\
Пусть $h := tu$, где $t \in \mathbb{R}$, $u \in \mathbb{R}^m$\\\\
Тогда по определению производной в точке $a$, для функции $F$:\\\\
$F(a + tu) = F(a) + L(tu) + o(tu)$\\\\
По свойству линейности линейного оператора: \\\\
$F(a + tu) = F(a) + t \cdot L(u) + o(t)$\\\\
Выражаем $L(u)$:\\\\
$L(u) = \dfrac{F(a + tu) - F(a)}{t} - \dfrac{o(t)}{t}$\\\\
$L(u) = \lim\limits_{t \rightarrow 0} \dfrac{F(a + tu) - F(a)}{t}$\\\\
В силу единственности предела - оператор тоже определен однозначно

\newpage


\subsection{Лемма о дифференцируемости отображения и его координатных функций}
Пусть $F : E \subset \mathbb{R}^m \rightarrow \mathbb{R}^l$, $F = (F_1, \ldots, F_l)$, $a \in \operatorname{Int} (E)$\\\\
Тогда \\\\
1. $F$ ~--- дифференцируема в точке $a$ $\Leftrightarrow$ все $F_i$ дифференцируемы в точке $a$\\\\
2. Строки матрицы Якоби $F$ равны матрицы Якоби функций $F_i$ \\\\\\
\textbf{Доказательство}        \\\\
1. В правую сторону:\\\\
Пусть $F$ дифференцируема в точке $a$. Тогда для каждой координатной функции должно выполняться следующее равенство:\\\\
$f_i(x + h) = f_i(x) + A_ih + \alpha_i(h) |h|$ для всех $i$\\\\
т.к. координатные функции линейного оператора $A$ являются линейными, а также непрерывность и равенство нулю в нуле отображения $\alpha$ равносильно такому же свойству его координатных функций, итого получили, что $f_i$ дифференцируема в точке $a_i$\\\\
В левую сторону:\\\\
Пусть все $f_i$ дифференцируемы в точке $a$. Тогда для каждого $i$ существует линейная функция $A_i$ и функция $\alpha_i$, непрерывная и равная нулю в нуле, для которых справедливо равенство. Значит для $f$ также выполняется равенство\\\\
$f(x + h) = f(x) + Ah + \alpha(h) |h|$\\\\\\
2. Распишем $F(x) = F(a) + F'(a)(x - a) + \alpha(x) |x - a|$ \\\\
$\begin{pmatrix} F_1(x) \\ F_2(x) \\ \vdots \\ F_l(x) \end{pmatrix} = \begin{pmatrix} F_1(a) \\ F_2(a) \\ \vdots \\ F_l(a) \end{pmatrix} + \begin{pmatrix} \lambda_{11} & \lambda_{12} & \ldots & \lambda_{1m} \\ \lambda_{21} & \lambda_{22} & \ldots & \lambda_{2m} \\ \ldots & \ldots & \ldots & \ldots \\ \lambda_{l1} & \lambda_{l2} & \ldots & \lambda_{lm} \end{pmatrix} \begin{pmatrix} x_1 - a_1 \\ x_2 - a_2 \\ \vdots \\ x_m - a_m \end{pmatrix} + \begin{pmatrix} \varphi_1(x) \\ \varphi_2(x) \\ \vdots \\ \varphi_m(x) \end{pmatrix} |x - a|$\\\\
Откуда и получаем требуемое условие
\newpage

\subsection{Необходимое условие дифференцируемости}
Пусть $f : E \subset \mathbb{R}^m \rightarrow \mathbb{R}$, $a \in$ Int $E$\\\\
$f$ ~--- дифференцируема в точке $a$\\\\
Тогда $\exists f'_{x_1}(a), \ldots, f'_{x_m}(a)$ и тогда $(f'_{x_1}(a), \ldots, f'_{x_m}(a))$ ~--- матрица Якоби $f$ в точке $a$\\\\
\textbf{Доказательство}\\\\
$f(a + h) = f(a) + L \cdot h + \alpha(h) \cdot |h|$\\\\
Пусть $t \in \mathbb{R}$, \\\\
$e_k = (0, 0, \ldots, 1, \ldots, 0)$, где $1$ находится на $k$-ом месте.\\\\
Тогда $h_k := t \cdot e_k$\\\\
$f(a + t \cdot e_k) = f(a) + l_k \cdot t + \alpha(t \cdot e_k) |t|$, где $l_k$ - строка матрицы Якоби функции $F$.\\\\
$l_k = \varphi'_k (a_k) = \dfrac{\partial f}{\partial_{x_k}} (a_k)$
\newpage
\subsection{Достаточное условие дифференцируемости}
Пусть $f : E \subset \mathbb{R}^m \rightarrow \mathbb{R}$, $a \in E$, $B(a, r) \subset E$\\\\
Пусть в этом шаре $\exists f'_{x_1} (x), \ldots, f'_{x_m} (x)$, $x \in B(a, r)$\\\\
и все эти производные непрерывны в точке $a$. Тогда $f$ ~--- дифференцируемы в точке $a$\\\\
\textbf{Доказательство}\\\\
Пусть $m = 2$, на большую размерность обобщается легко\\\\
$a = (a_1, a_2)$, $x = (x_1, x_2)$\\\\
$f(x_1, x_2) - f(a_1, a_2) = \left( f(x_1, x_2) - f(a_1, x_2) \right) + \left(f(a_1, x_2) - f(a_1, a_2) \right) =$\\\\ 
По теореме Лагранжа $f(b) - f(a) = f'(c) (b - a)$, значит\\\\
$= f'_{x_1} (\overline{x_1}, x_2)(x_1 - a_1) + f'_{x_2}(x_1, \overline{x_2})(x_2 - a_2) = f'_{x_1}(a_1, a_2)(x_1 - a_1) + f'_{x_2} (a_1, a_2)(x_2 - a_2) + (f'_{x_1}(\overline{x_1}, x_2) - f'_{x_1}(a_1, a_2))(x_1 - a_1) + (f'_{x_2}(a, \overline{x_2}) - f'_{x_2}(a_1, a_2))(x_2 - a_2) \rightarrow 0$ при $(x_1, x_2) \rightarrow (a_1, a_2)$, поскольку\\\\
Пусть $\alpha(h) |h| = (f'_{x_1}(\overline{x_1}, x_2) - f'_{x_1}(a_1, a_2))(x_1 - a_1) + (f'_{x_2}(a, \overline{x_2}) - f'_{x_2}(a_1, a_2))(x_2 - a_2)$, где $|h| = \sqrt{(x_1 - a_1)^2 + (x_2 - a_2)^2}$\\\\
Тогда в качестве примера рассмотрим первое слагаемое $(f'_{x_1}(\overline{x_1}, x_2) - f'_{x_1}(a_1, a_2)) \cdot \dfrac{x_1 - a_1}{|h|}$, которое стремится к нулю, поскольку первый множитель стремится к нулю при $(x_1, x_2) \rightarrow (a_1, a_2)$, а второй множитель не превосходит по модулю $1$.


\newpage
\subsection{Метод Лапласа}
\newpage

\end{document}