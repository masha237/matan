\documentclass[../main.tex]{subfiles}
\graphicspath{{\subfix{../Images/}}}
\begin{document}

\lhead{Волков}
\subsection{Дифференциальный критерий выпуклости}
Ссылка:
https://youtu.be/Ddj4g9BI0d4?t=6293 \\ \\
Сама теорема: \\ 
1. $f \in C<a, b>$, $f$ - дифференциируема на $(a, b)$, тогда \\
\indent $f$ (строго) выпукла $\Leftrightarrow f'$ (строго) возрастает на $(a, b)$ \\ \\
2. $f \in C<a, b>$, $f$ - дважды дифференциируема на $(a, b)$, тогда \\
\indent $f$ - выпуклаяы $\Leftrightarrow f'' \geq 0$ на $(a, b)$ \\ \\
Доказательство: \\
1 $\Rightarrow: x_1 < x_2, f'(x_1) = f_{+}' (x_1) \leq f'_{-}(x_2) = f'(x_2)$  \\
1 $\Leftarrow: x_1 < x < x_2$ , $\frac{f(x) - f(x_1)}{x - x_1} = f'(c_1)$  $ \leq f'(c_2)= \frac{f(x_2) - f(x)}{x_2 - x}$ (Лемма о 3-х хордах) \\
2 $f$ - выпукла $\Leftrightarrow$ $f$ - возрастает (доказанно в пункте 1) $\Leftrightarrow$ $f'' \geq 0$
\newpage

\lhead{А кто?}
\subsection{Теорема о вычислении аддитивной функции промежутка по плотности}

        \subsection*{Формулировка}
        
            Пусть заданы $f$ и $\phi$ на $\langle a, b \rangle$, $f$ ~--- непрерывна, $\phi$ ~--- аддитивная функция промежутка, $f$ ~--- плотность $\phi$
            
            Тогда $\forall [p, q] \subset \langle a, b \rangle \ \phi([p, q]) = \int\limits^q_p f(x) \ dx$
            
        \subsection*{Доказательство}
        
            Можно принять за факт, что у нас дан промежуток $[a, b]$ (если это не так, то уменьшим его чуть-чуть и переобозначим)
            
            $F(x) = \begin{cases} 0{,} & x = a \\ \phi([a, x]){,} & x > a \end{cases}$ ~--- первообразная $f$
            
            $\inf\limits_{[x, x + h]} f \leq \frac{\phi([x, x + h])}{h} \leq \sup\limits_{[x, x + h]} f$
            
            $x : F'_+(x) = \lim\limits_{h \rightarrow 0 + 0} \frac{F(x + h) - F(x)}{h} = \lim \frac{\phi([a, x + h]) - \phi([a, x])}{h} = \lim \frac{\phi([x, x + h])}{h} = \lim\limits_{h \rightarrow 0 + 0} f(x + \Theta h) = f(x)$, где
            
            $0 \leq \Theta \leq 1$
            
            $\Theta = \Theta(h)$
            
            Аналогично посчитаем и $F'_-(x)$
            
            $\phi([p, q]) = F(q) - F(p) = \int\limits^q_p f(x) \ dx$
\newpage


\subsection{Площадь криволинейного сектора: в полярных координатах и для параметрической кривой}
\subsection*{Формулировка для полярных координат}
            
            Пусть $[\alpha, \beta] \subset [0, 2 \pi)$
            
            $\rho : [\alpha, \beta] \rightarrow \mathbb{R}$ ~--- непрерывная, $\rho \geq 0$
            
            $A = \left\{ (r, \phi) : \phi \in [\alpha, \beta] \ 0 \leq r \leq \rho(\phi) \right\}$ ~--- <<Аналог ПГ>>
            
            Тогда $\sigma(A) = \frac{1}{2} \int\limits^{\beta}_{\alpha} \rho^2(\phi) \ d \phi$
            
        \subsection*{Доказательство}
        
            $[\alpha, \beta] \longmapsto \sigma(A)$ ~--- функция промежутка $\operatorname{Segm} [\alpha, \beta]$ ~--- аддитивная функция.
            
            Проверим, что $\frac{1}{2} \rho^2 (\phi)$ ~--- плотность
            
            $[\gamma, \delta]$ ~--- строим $A_{\gamma{,} \delta}$
            
            $\sigma (A_{\gamma{,} \delta}) \leq \sigma$(Круговой сектор $(0, \max\limits_{[\gamma, \delta]} \rho(\phi), [\gamma, \delta])$)
            
            $\sigma (A_{\gamma{,} \delta}) \geq \sigma$(Круговой сектор $(0, \min\limits_{[\gamma, \delta]} \rho(\phi), [\gamma, \delta])$)
            
            $\min\limits_{[\gamma, \delta]} \frac{1}{2} \rho^2(\phi) l([\gamma, \delta]) \leq \sigma (A_{\gamma{,} \delta}) \leq \max\limits_{[\gamma, \delta]} \frac{1}{2} \rho^2(\phi) l([\gamma, \delta])$
            
            По определению плотности получили то, что хотели
            
            (если непонятно, откуда берётся $\dfrac{1}{2} \rho^2(\phi)$, то гуглим форму площади круга)
            
        \subsection*{Формулировка для параметрической кривой}
        
            $\sigma(A) = \dfrac{1}{2} \int\limits^{t_1}_{t_0} (y'(t) x(t) - x'(t) y(t)) dt$
            
        \subsection*{Доказательство}
        
            Пусть дано $(x(t), y(t)), t \in [a, b]$
            
            $x = r \cos {\varphi}$
            
            $y = r \sin {\varphi}$
            
            $r = \sqrt{x^2 + y^2}$
            
            $\varphi = \arctg \dfrac{y}{x}$
            
            Итого:
            
            $r(t) = x(t)^2 + y(t)^2$ и $\varphi(t) = \arctg \dfrac{y(t)}{x(t)}$ ~--- параметрическое задание того же пути в полярных координатах
            
            $\sigma (A) = \dfrac{1}{2} \int\limits^{\varphi_1}_{\varphi_0} r^2(\varphi) d \varphi = \dfrac{1}{2} \int\limits^{\varphi_1}_{\varphi_0} \left(x(t)^2 + y(t)^2 \right) \left( \arctg \dfrac{y(t)}{x(t)} \right) dt$
            
            $\dfrac{1}{2} \int\limits^{t_1}_{t_0} (x^2 + y^2) \dfrac{1}{1 + \frac{y^2}{x^2}} \cdot \dfrac{y'x - x'y}{x^2} dt = \dfrac{1}{2} \int\limits^{t_1}_{t_0} \left( y'(t) x(t) - x'(t) y(t) \right) dt$
            
\newpage


\subsection{Изопериметрическое неравенство}
 \subsection*{Формулировка}
        
            Пусть $G$ ~--- замкнутая выпуклая фигура в $\mathbb{R}^2$
            
            $\operatorname{diam} G \leq 1$ ($\operatorname{diam} G = \sup\limits_{x, y \in G} \rho(x, y)$)
            
            Тогда $\sigma(G) \leq \frac{\pi}{4}$
            
        \subsection*{Доказательство}
        
            Поскольку $G$ выпукла, значит к ней можно провести касательные $f(x)$ и $g(x)$
            
            $f(x) = \sup \left\{ t : [(x, 0), (x, t)] \cap G = \varnothing \right\}$
            
            $g(x)$ ~--- аналогично, только $\inf$
            
            $\varphi \in \left[ -\frac{\pi}{2}, \frac{\pi}{2} \right]$
            
            $r(-\frac{\pi}{2}) = r(\frac{\pi}{2}) = 0$
            
            $r(\varphi)$ ~--- непрерывная функция от $\varphi$
            
            $\sigma(G) = \frac{1}{2} \int\limits^{\pi / 2}_{-\pi / 2} r^2 (\varphi) \ d \varphi = \frac{1}{2} \left( \int\limits^0_{-\pi / 2} + \int\limits^{\pi / 2}_0 \right)$
            
            Проведём какую-нибудь прямую $AB$, полностью лежащую в фигуре $G$, а также отметим какую-нибудь точку $O$, что $OA \perp OB$. Тогда
            
            $\sigma(G) = \frac{1}{2} \int\limits^{\pi / 2}_0 OA^2 + OB^2$
            
            $\sigma(G) = \frac{1}{2} \int\limits^{\pi / 2}_0 r^2 (\varphi - \dfrac{\pi}{2}) + r^2 (\varphi) \ d \varphi = \frac{1}{2} \int\limits^{\pi / 2}_0 AB^2 \ d \varphi \leq \frac{1}{2} \int\limits^{\pi / 2}_0 1 \ d \varphi = \dfrac{\pi}{4}$
            
\newpage


\subsection{Обобщенная теорема о плотности}
\subsection*{Формулировка}
            
            Пусть $f : \langle a, b \rangle \rightarrow \mathbb{R}$ ~--- непрерывная функция, $\phi : \operatorname{Segm} \langle a, b \rangle \rightarrow \mathbb{R}$ ~--- аддитивная функция.
        
            Пусть $\forall \Delta \subset \operatorname{Segm} \langle a, b \rangle$ заданы числа $m_{\Delta}$, $M_{\Delta}$.
        
            \begin{enumerate}
        
                \item $m_{\Delta} \cdot l(\Delta) \leq \phi(\Delta) \leq M_{\Delta} \cdot l(\Delta)$
            
                \item $\forall x \in \Delta$ $m_{\Delta} \leq f(x) \leq M_{\Delta}$
            
                \item $\forall x \in \langle a, b \rangle$ $M_{\Delta} - m_{\Delta} \rightarrow 0$, если $l(\Delta) \rightarrow 0$, $x \in \Delta$
            
            \end{enumerate}
            
            3-й пункт можно переформулировать по-другому:
            
            $\forall \varepsilon > 0 : \exists \delta > 0 : \forall \Delta \in \operatorname{Segm} \langle a, b \rangle : x \in \Delta : l(\Delta) < \delta \Longrightarrow \left| M_{\Delta} - m_{\Delta} \right| < \varepsilon$
        
            Тогда $f$ ~--- плотность $\phi$ (и $\forall [p, q]$ $\phi([p, q]) = \int\limits^q_p f(x) \ dx$)
        
        \subsection*{Доказательство}
        
            $F(x) = \begin{cases} 0, & x = 0 \\ \phi([a, x]), & x > a \end{cases}$
        
            Дифференцируем $F_+$
        
            $m_{\Delta} \leq \frac{F(x+ h) - F(x)}{h} \leq M_{\Delta}$
        
            $\left| \frac{F(x + h) - F(x)}{n} - f(x) \right| \leq \left| M_{\Delta} - m_{\Delta} \right| \xrightarrow[h \rightarrow 0]{} 0$, $\Delta = [x, x + h]$
        
            $\frac{F(x + h) - F(x)}{h} \xrightarrow[h \rightarrow 0]{} f(x)$
        
            Аналогично и с $F_-$
\newpage


\end{document}