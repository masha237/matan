\documentclass[../main.tex]{subfiles}
\graphicspath{{\subfix{../Images/}}}
\begin{document}


\subsection{Техническое описание верхнего предела}
 \subsection*{Формулировка}
        
            \begin{enumerate}
            
                \item $\varlimsup x_n = +\infty \Longleftrightarrow x_n$ ~--- не ограничена сверху
                
                \item $\varlimsup x_n = -\infty \Longleftrightarrow x_n \rightarrow -\infty$
                
                \item $\varlimsup x_n = l \in \mathbb{R} \Longrightarrow$:
                
                \begin{itemize}
                
                    \item $\forall \varepsilon > 0 : \exists N: \forall n > N \ \ \ x_n < l + \varepsilon$
                    
                    \item $\forall \varepsilon > 0$ неравенство $x_n > l - \varepsilon$ выполняется для бесконечного множества номеров $n$
                
                \end{itemize}
                
            \end{enumerate}
            
        \subsection*{Доказательство}
        
            \begin{enumerate}
            
                \item Очевидно, что $x_n < y_n$, $y_n$ убывает Таким образом, если $\lim y_n = +\infty \Longrightarrow y_n = +\infty \Longleftrightarrow x_n$ ~--- не ограничена сверху
                
                \item $y_n \rightarrow -\infty$, $\forall E : \exists N : \forall n > N \ x_n \leq y_n < E \Rightarrow \forall E > 0 : \exists N : \forall n > N : x_n < E, y_n < E$
                
                \item $x_n \leq y_n$, $y_n \rightarrow l$
                
                \begin{itemize}
                
                    \item $\Rightarrow)$ $\forall \varepsilon > 0 : \exists N : \forall n > N : x_n \leq y_n < l + \varepsilon$
                    
                        Если $\exists N_0 : \forall n > N_0 : x_n < l - \varepsilon$, то $\forall n > N_0 : y_n = \sup(\ldots) \leq l - \varepsilon$ и тогда $y_n \nrightarrow l$
                        
                    \item $\Leftarrow)$ $\forall \varepsilon : \exists N : \forall n > N : y_n \leq l + \varepsilon$, $y_n$ ~--- супремум
                    
                        $x_k \geq l - \varepsilon \Rightarrow y_n \geq l - \varepsilon \Rightarrow y_n \rightarrow l$
                        
                \end{itemize}
                
            \end{enumerate}
\newpage


\subsection{Теорема о существовании предела в терминах верхнего и нижнего пределов}
\subsection*{Формулировка}
        
            Пусть существует $\lim x_n = l \in \overline{\mathbb{R}}$, тогда и только тогда $\varlimsup x_n = \varliminf x_n = l$
            
        \subsection*{Доказательство}
        
            \begin{itemize}
            
                \item $\Rightarrow)$ $\lim x_n = +\infty \Longleftrightarrow \varliminf x_n = +\infty \Rightarrow \varliminf \leq \varlimsup x_n = +\infty$
                
                    $\lim x_n = -\infty \Longleftrightarrow \varliminf x_n \leq \varlimsup = -\infty$
                    
                    $\lim x_n \in \mathbb{R}$ ~--- очевидно
                    
                \item $\Leftarrow)$ $z_n \leq x_n \leq y_n$, то по теореме о сжатой последовательности $x_n \rightarrow l$, поскольку $z_n \rightarrow l$ и $y_n \rightarrow l$
                
            \end{itemize}
\newpage


\subsection{Теорема о характеризации верхнего предела как частичного}
\subsection*{Формулировка}
            
            \begin{enumerate}
            
                \item Пусть $l$ ~--- частный предел $x_n$, тогда $\varliminf x_n \leq l \leq \varlimsup x_n$
                
                \item Существуют такие $n_k$, $m_k$, что $\lim x_{n_k} = \varlimsup x_n$ и $\lim x_{m_k} = \varliminf x_n$
                
            \end{enumerate}
            
        \subsection*{Доказательство}
        
            \begin{enumerate}
            
                \item Пусть $x_{n_j} \rightarrow l$
                
                    $z_{n_j} \leq x_{n_j} \leq y_{n_j}$, где $z_{n_j} \rightarrow \varliminf x_n$, $x_{n_j} \rightarrow l$, $y_{n_j} \rightarrow \varlimsup x_n$
                    
                \item $\varlimsup x_k = \pm \infty$ ~--- очевидно
                
                    $\varlimsup x_k = l \in \mathbb{R}$ ~--- очевидно
                    
                    Для $\varepsilon = \frac{1}{k}$ $\exists x_{n_k} : l - \frac{1}{k} \leq x_{n_k} \leq l + \frac{1}{k}$
                    
            \end{enumerate}
\newpage


\subsection{Асимптотика степенных сумм}
$f(x) = x^p$, $p \neq -1$
        
        $1^p + 2^p + \ldots + n^p = \int\limits^n_1 x^p dx + \dfrac{n^p + 1}{2} + \dfrac{1}{2} \int\limits^n_1 (x^p)'' \left\{ x \right\} (1 - \left\{ x \right\}) dx$
            
        $1^p + 2^p + \ldots + n^p = \dfrac{n^{p + 1}}{p + 1} - \dfrac{1^{p + 1}}{p + 1} + \dfrac{n^p}{2} + \dfrac{1}{2} + \dfrac{p(p - 1)}{2} \int\limits^n_1 x^{p - 2} \left\{ x \right\} (1 - \left\{ x \right\})$
            
        $1^p + 2^p + \ldots + n^p = \dfrac{n^{p + 1}}{p + 1} + \dfrac{n^p}{2} + \mathcal{O}(\max(1, n^{p - 1}))$
\newpage


\subsection{Асимптотика частичных сумм гармонического ряда}

        $1 + \dfrac{1}{2} + \ldots + \dfrac{1}{n} = \int\limits^n_1 \dfrac{1}{x} dx + \dfrac{1}{2} + \dfrac{1}{2n} + \int\limits^n_1 \dfrac{1}{x^3} \left\{ x \right\}(1 - \left\{ x \right\}) dx$
        
        $1 + \dfrac{1}{2} + \ldots + \dfrac{1}{n} = \ln n + \dfrac{1}{2} + \dfrac{1}{2n} + \int\limits^n_1 \dfrac{\left\{ x \right\}(1 - \left\{ x \right\})}{x^3} dx$
        
        Интеграл постоянной возрастает и ограничен сверху $\dfrac{1}{4} \int\limits^n_1 \dfrac{1}{x^3} dx = -\dfrac{1}{x^2} \cdot \dfrac{1}{8} \bigg|^{x = n}_{x = 1} < \dfrac{1}{8}$
        
        Всё, что правее логарифма ~--- постоянная Эйлера или $\gamma$
        
        Итого
        
        $1 + \dfrac{1}{2} + \ldots + \dfrac{1}{n} = \ln {n} + \gamma + o(1)$
\newpage


\end{document}