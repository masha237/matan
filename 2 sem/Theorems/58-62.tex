\documentclass[../main.tex]{subfiles}
\graphicspath{{\subfix{../Images/}}}
\begin{document}


\subsection{Признаки Дирихле и Абеля сходимости числового ряда}
 \subsection*{Формулировка}
        
            \subsubsection*{Дирихле}
            
                Пусть $S^{(a)}_n$ ~--- ограничена
                
                $b_n$ ~--- монотонна. $b_n \rightarrow 0$
                
                Тогда $\sum\limits^{+\infty}_{k = 1} a_k b_k$ ~--- сходится
                
            \subsubsection*{Абеля}
            
                Пусть $\sum a_k$ ~--- сходится, $b_n$ ~--- ограниченная, монотонная
                
                Тогда $\sum\limits^{+\infty}_{k = 1} a_k b_k$ ~--- сходится 
                
        \subsection*{Доказательство}
        
            \subsubsection*{Дирихле}
            
                Применим преобразование Абеля $\sum\limits^n_{k = 1} a_k b_k = A_n b_n + \sum\limits^{n - 1}_{k = 1} A_k (b_k - b_{k + 1})$
                
                Из того, что $A_n$ ограничена, а $b_n$ бесконечна мала, следует, что $A_n b_n \rightarrow 0$, поэтому сходимость эквивалентна сходимости ряда $\sum\limits^{\infty}_{k = 1} A_k (b_k - b_{k + 1})$
                
                $\sum\limits^{n - 1}_{k = 1} \left| A_k (b_k - b_{k + 1}) \right| \leq c_a \sum\limits^{n - 1}_{k = 1} \left| b_k - b_{k + 1} \right| = c_a |b_1 - b_n|$ ~--- ограничена
                
            \subsubsection*{Абеля}
            
                Существует конечный $\lim\limits_{n \rightarrow +\infty} b_n = \beta$
                
                $\sum a_k b_k = \sum a_k \beta + \sum a_k (b_k - \beta)$, первое сходится в силу сходимость $\sum a_k$, а второе сходится в силу признака Дирихле
                
\newpage


\subsection{Теорема о группировке слагаемых}
 \subsection*{Формулировка}
        
            Выберем $n_0 = 0 < n_1 < n_2 < \ldots$
            
            Пусть $\sum a_k = (a_1 + a_2 + \ldots + a_{n_1}) + (a_{n_1 + 1} + \ldots + a_{n_2}) + \ldots$
            
            $b_k = \sum\limits^{n_k}_{i = n_{k - 1} + 1} a_i$
            
            Тогда
            
            \begin{enumerate}
            
                \item $\sum a_n$ ~--- сходится $\Rightarrow \sum b_k$ сходится и имеет ту же сумму
                
                \item $a_k \geq 0 \Rightarrow \sum a_k = \sum b_k$ 
                
            \end{enumerate}
            
        \subsection*{Доказательство} 
            
            $S^{(b)}_k = S^{(a)}_{n_k}$
            
            \begin{enumerate}
            
                \item $\lim\limits_{k \rightarrow \infty} S^{(b)}_k = \lim\limits_{k \rightarrow \infty} S^{(a)}_{n_k} = S^{(a)}$
                
                \item Если $\sum a_n$ ~--- сходится, то смотри пункт $1$
                
                    Если $\sum a_n$ ~--- расходится, значит $S^{(a)}_n$ не ограничено сверху, значит и $S^{(b)}_n$ не ограничено сверху
            
            \end{enumerate}
\newpage


\subsection{Теорема о перестановке слагаемых}
\subsection*{Формулировка}
        
            \begin{enumerate}
            
                \item Пусть ряд $\sum a_n$ абсолютно сходится, тогда ряд $\sum b_n$, полученный из ряда $\sum a_n$ перестановкой, будет также абсолютно сходиться и иметь ту же сумму.
            
                \item Также если $a_k \geq 0$ при всех $k$, то $\sum a_k = \sum b_k$
                
            \end{enumerate}
            
        \subsection*{Доказательство}
        
            \begin{enumerate}
            
                \item По определению $S_n^{(b)} = a_{\varphi(1)} + \ldots + a_{\varphi(n)} \leq S^{(a)}_{\max \varphi(i)}$. Устремим $n \rightarrow +\infty$, $S^{(b)} \leq S^{(a)}$. Аналогично $S^{(a)} \leq S^{(b)}$. 
            
                \item Берём срезки $a_n^+$ и $a_n^-$, тогда $\sum a_n^+$, $\sum a_n^-$ ~--- сходятся.
            
                    $a^+_n = \max(a^+_n, 0)$, $\sum b^+_n$ ~--- перестановка ряда $a^+_n$
            
                    $a^-_n = \max(-a^-_n, 0)$. Аналогично $\sum b^-_n$
                
                    И по второму пункту всё доказали
                    
            \end{enumerate}
            
            P.S. доказательство идёт в обратном порядке
            
\newpage


\subsection{Теорема о произведении рядов}

        \subsection*{Формулировка}
        
            Пусть ряды $(A)$ и $(B)$ абсолютно сходятся к суммам $S^{(a)}$ и $S^{(b)}$. Тогда $\forall \gamma : \mathbb{N} \rightarrow \mathbb{N} \times \mathbb{N}$ ~--- биекция, произведение рядов абсолютно сходится и имеют сумму $S^{(a)} S^{(b)}$
            
        \subsection*{Доказательство}
        
            Пусть $\sum |a_k| = A$ и $\sum |b_k| = B$, тогда
            
            $\sum\limits^N_{k = 1} |a_{\varphi(k)} b_{\psi(k)}| \leq \sum\limits^n_{k = 1} |a_n| \sum\limits^m_{k = 1} |b_k| \leq A \cdot B$, где $n := \max(\varphi(1), \ldots, \varphi(N))$, $m = \max(\psi(1), \ldots, \psi(N))$
            
            Значит ряд $\sum\limits^N_{k = 1} |a_{\varphi(k)} b_{\psi(k)}|$ ~--- сходится, значит произведение рядов абсолютно сходится, значит его сумма не зависит от порядка слагаемых, следовательно не зависит и от выбора $\gamma$
            
\newpage


\subsection{Теорема об условиях сходимости бесконечного произведения}
\subsection*{Формулировка}
        
            \begin{enumerate}
            
                \item Пусть $a_n > 0$ НСНМ. Тогда равносильность $\prod\limits (1 + a_n)$ ~--- сходится $\Longleftrightarrow \sum a_n$ ~--- сходится
                
                \item Пусть $\sum a_n$ ~--- сходится, а также $\sum a^2_n$ ~--- тоже сходится. Тогда $\prod (1 + a_n)$ ~--- сходится
                
            \end{enumerate}
        
        \subsection*{Доказательство}
        
            \begin{enumerate}
            
                \item $\prod$ ~--- сходится $\Leftrightarrow \sum \ln |1 + a_n| $~--- сходится $\Leftrightarrow \sum a_n$ ~--- сходится. НСНМ $\ln |1 + a_n| \sim a_n$ при $n \rightarrow +\infty$
                
                \item $\prod$ сходится $\Leftrightarrow \sum \ln (1 + a_n)$ сходится
                
                    $\ln(1 + a_n) = a_n - \dfrac{a^2_n}{2} + o(a^2_n)$
                    
                    Докажем, что $\sum |o(a^2_n)|$ абсолютно сходится
                    
                    $\lim\limits_{n \rightarrow} \dfrac{o(a^2_n)}{a^2_n} = 0$ из сходимости $\sum a^2_n$ следует сходимость $\sum |o(a^2_n)|$, значит и $\sum o(a^2_n)$ сходится
                
            \end{enumerate}
\newpage


\end{document}