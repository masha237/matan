\documentclass[../main.tex]{subfiles}
\graphicspath{{\subfix{../Images/}}}
\begin{document}


\subsection{Гладкий путь, вектор скорости, носитель пути}
~

\textbf{Путь}~--- непрерывное отображение $\gamma:[a,b]\to\R^m$.

$\gamma(a)$~--- начало, $\gamma(b)$~--- конец.

Можем представить путь в виде его координатных функций $\gamma(t)\mapsto(\gamma_1(t), \dots, \gamma_m(t))$.

\textbf{Гладкий путь}:
\begin{equation*}
    \forall i \in \{1,\dots,m\} \quad \gamma_i\in C^1 [a,b]
\end{equation*}

\textbf{Вектор скорости}:
\begin{equation*}
    \gamma'(t)=\lim_{h \to 0}\frac{\gamma(t+h)-\gamma(t)}{h}
\end{equation*}

\textbf{Носитель пути}~--- множество всех значений $\gamma$ на $[a,b]$, обозначется $C_\gamma$. 

\subsection{Длина гладкого пути}
\textbf{Длина пути}~--- фнукция $l:$ множество всех гладких путей $\R^m$ $\mapsto [0, \infty)$, такая что:
\begin{enumerate}
    \item \textit{Аддитивность}: $\forall \gamma:[a,b]\to\R^m \quad \forall c\in(a,b) \quad l(\gamma)=l(\gamma|_{[a,c]})+l(\gamma|_{[c,b]})$
    \item $\forall \gamma_1, \gamma_2$~--- гладкие, если существует сжатие $T:C_{\gamma_1}\to C_{\gamma_2}$, то $l(\gamma_2)\le l(\gamma_1)$.\\\\
    Сжатие~--- это такое $T:A\to B$, что $\forall a_1, a_2\in A \quad \rho(T(a_1),T(a_2))\le\rho(a_1, a_2)$
    \item Линейный путь $\gamma:t\mapsto(tv+u) \Rightarrow l(\gamma)=\rho(\gamma(a), \gamma(b))$, где $u,v\in\R^m$
\end{enumerate}

\subsection{Вариация функции на промежутке}
https://youtu.be/TWcK0dCUw98?t=53m25s

\begin{gather*}
    f:[a,b]\to\R \\
    \operatornamewithlimits{Var}_a^b(f)=\sup_n\sum_{i=1}^{n}|f(x_i)-f(x_{i-1})|:\quad a=x_0<x_1<\dots<x_n=b\\
    f\in C^1[a,b]\Rightarrow\operatornamewithlimits{Var}_a^b(f)=\int_a^b |f'(x)|dx
\end{gather*}

\subsection{Дробление отрезка, ранг дробления, оснащение}
https://youtu.be/TWcK0dCUw98?t=1h10m43s

\textbf{Дробление отрезка} $[a,b]$~--- набор точек $\{x_i\}_{i=0}^{n}:\quad a=x_0<x_1<\dots<x_n=b$.

\textbf{Ранг} дробления~--- $\max\limits_{i=1}^{n}(x_i-x_{i-1})$.

\textbf{Оснащение} дробления~--- множество точек $\{\xi_i\}_{i=0}^{n}:\quad\forall i\in\{1,\dots,n\}\quad\xi_i\in[x_{i-1}, x_i]$.


\subsection{Риманова сумма}
\textbf{Интегральная \textit{(риманова)} сумма} для разбиения $\{x_i\}$, произвольной функции $f$ и оснащения $\{\xi_i\}$ это следующая сумма:
    $$\sum_{i=1}^n f(\xi_i)(x_i-x_{i-1})$$

\newpage
\end{document}