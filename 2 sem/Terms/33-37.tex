\documentclass[../main.tex]{subfiles}
\graphicspath{{\subfix{../Images/}}}
\begin{document}


\subsection{Критерий Больцано--Коши сходимости числового ряда}
\begin{equation*}
    \sum a_n \text{ сходится }\Longleftrightarrow \forall \varepsilon > 0 \quad \exists N: \ 
    \forall k > N \quad \forall m \in \N \quad \left| a_{k+1} + a_{k+2} + \dots + a_{k+m} \right| < \varepsilon 
\end{equation*}

\subsection{Преобразование Абеля}
https://youtu.be/1Y8wiAt9\_ik?t=33m18s
\begin{equation*}
    \sum_{k=1}^{n}a_k b_k = A_n b_n + \sum_{k=1}^{n-1}A_k(b_k - b_{k+1}) \qquad \left(A_m=\sum_{k=1}^{m}a_k\right)
\end{equation*}
Доказательство тривиально: можно легко убедится, что все слагаемые $a_i b_j$ входят в обе части одинаковое количество раз.
\footnote{Само преобразование чем-то напоминает $\int f'g = fg - \int fg'$ но только для сумм рядов, а не интегралов функций.
То есть тут $A_m$~--- что-то в духе первообразной, а $(b_{k+1} - b_k)$~--- производной.
Только, грубо говоря, дано у нас это в виде $\int fg = Fg + \int f(-g')$.}


\subsection{Произведение рядов}
https://youtu.be/1Y8wiAt9\_ik?t=2h41m50s \\
$\sum a_k, \sum b_k, \qquad \gamma: \N \to \N \times \N$~--- биекция, $\gamma(k)=\left(\varphi(k),\psi(k)\right)$ \\
\textbf{Произведение рядов} $A$ и $B$~--- ряд $\sum\limits_{k=1}^{\infty} a_{\varphi(k)} b_{\psi(k)}$


\subsection{Произведение степенных рядов}
https://youtu.be/bJiRRCj630Q?t=0s\\
$x\in\R$, $x$~--- фиксированный \\
Произведением степенных рядов $\sum\limits_{k=0}^{\infty}a_k x^k$ и $\sum\limits_{k=0}^{\infty}b_k x^k$
называется ряд $\sum\limits_{k=0}^{\infty}c_k x^k$,\\где $c_k=\sum\limits_{i=0}^{k} a_{i}b_{k-i}$

\newpage
\subsection{Бесконечное произведение}
https://youtu.be/bJiRRCj630Q?t=9m19s

Бескноечное произведение $\prod\limits_{i=1}^{\infty} p_i$, где $p_i \in \R$.
Введём обозначение для частичного произведения $\pi_N := \prod\limits_{i=1}^{N} p_i$.
Пусть $\lim\limits_{N\to\infty}\pi_N=P$.
Определим $\prod\limits_{i=1}^{\infty} p_i$ следующим образом:
\begin{itemize}
    \item $P \in (0, +\infty) \Leftrightarrow \prod\limits_{i=1}^{\infty} p_i$ сходится к $P$
    \item $P = +\infty \Leftrightarrow \prod\limits_{i=1}^{\infty} p_i$ расходится к $+\infty$
    \item $P = 0 \Leftrightarrow \prod\limits_{i=1}^{\infty} p_i$ расходится к $0$
    \item $\nexists \lim\limits_{N\to\infty}\pi_N \Leftrightarrow \prod\limits_{i=1}^{\infty} p_i$ расходится
\end{itemize}


\end{document}