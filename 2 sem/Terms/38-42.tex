\documentclass[../main.tex]{subfiles}
\graphicspath{{\subfix{../Images/}}}
\begin{document}

\subsection{Скалярное произведение, евклидова норма и метрика в $R^m$}
https://youtu.be/bJiRRCj630Q?t=2h08m40s
\begin{gather*}
    \textbf{Скалярное произведение: } \left<x,y\right>=\sum x_i y_i \\
    \textbf{Евклидова норма: } \left\|x\right\|=\sqrt{\left<x,x\right>} \\
    \textbf{Метрика в $\R^m$: } \rho(x,y)=\left\|x-y\right\|
\end{gather*}
Вообще норма бывает и не только евклидовой
(подробнее о том, что такое норма, смотри линал за второй сем или матан за первый).
Кокнретно евклидову норму будем обозначать как $|x|$, но ни в коем случае нельзя называть её модулем.

\subsection{Окрестность точки в $R^m$, открытое множество}
https://youtu.be/bJiRRCj630Q?t=2h15m

\textbf{Окрестность} точки $x$ (радиуса $r$)~--- открытый шар.
\begin{equation*}
    U(x)=U_r(x)=\left\{y \in \R^m: |y-x|<r \right\}
\end{equation*}

Точка $x \in A$ является внутренней $\Leftrightarrow \exists U(x) \subset A$.

\textbf{Открытое множество}~--- множество, все точки которого, являются внутренними.

\newpage
\subsection{Сходимость последовательности в $R^m$, покоординатная сходимость}

Пара определений того, что значит фраза:
<<Последовательность~$x_n$~в~$R^m$~сходится~(к~$a$)>>
\begin{itemize}
    \item $\lim\limits_{n\to\infty}|x_n-a|=0$
    \item $\forall U(a)\quad \exists N:\ \forall n>N\quad x_n \in U(a)$
\end{itemize}
Похоже на $\R^1$, не правда ли?
Лишь норма вместо модуля (забавный факт: модуль~--- евклидова норма в $\R^1$)
и обобщённое определение окрестности.
Если вдруг спросят сходимость функций $\R^m\to\R^n$, то там опять всё похоже на $\R\to\R$.

\textbf{Покоординатная сходимость}:
$x_n\to a \Leftrightarrow \forall k \in \{1,\dots,m\} \quad (x_n)_k \to a_k$\\
Аналогично для функций.

\subsection{Предельная точка, замкнутое множество, замыкание}
https://youtu.be/bJiRRCj630Q?t=2h16m40s

$\dot{U}(x)=U(x)\setminus\{x\}$

$x$ \textbf{предельная точка} $A$ $\Leftrightarrow \forall U(x) \quad \dot{U}(x) \cap A \neq \varnothing$.\\
Замечание: предельная точка может и не принадлежать множеству
(напрмиер, $1$~--- предельная точка $[0,1)$) 

\textbf{Замкнутое множество}~--- множество, содержащее все свои предельные точки.

\textbf{Замыкание} множества~--- объединение множества со всеми его предельными точками.

\subsection{Компактность, секвенциальная компактность, принцип выбора Больцано-Вейерштрасса}
https://youtu.be/bJiRRCj630Q?t=2h19m45s

Интересный факт: объединение любого семейства открытых множеств всегда открыто
и пересечение конечного семейства открытых множеств тоже всегда открыто.

\textbf{Покрытие} множества $K$~--- такое семейство множеств $\{G_a\}_{a \in A}$,
что $K \subset \bigcup\limits_{a \in A}G_a$.

\textbf{Открытое покрытие}~--- покрытие семейством открытых множеств.

В общем случае, \textbf{компактное множетсво}~--- множество, из любого открытого покрытия которого
можно извлечь конечное подпокрытие.

Для метрических множеств \textbf{компактность} равносильна одновременной замкнутости и ограниченности.

Из компактности следует \textbf{секвенциальная компактность}~---
возможность из любой последовательности в этом множестве выделить сходящуюся подпоследовательность.

\textbf{Принцип выбора Больцано-Вейерштрасса}: из всякой ограниченной последовательности в $\R^m$
можно выделить сходящуюся подпоследовательность.
\end{document}