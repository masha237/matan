\documentclass[../main.tex]{subfiles}
\graphicspath{{\subfix{../Images/}}}
\begin{document}


\subsection{Положительная и отрицательная срезки}\label{subsec:1.8}

$f$ - функция.

$f+ = max(f, 0)$ - положительная срезка.

$f- = max(-f, 0)$ - отрицательная срезка.



\subsection{Среднее значение функции на промежутке}

$f \in C[a, b]$, тогда $\frac{1}{b-a} \int_{a}^{b} f \,dx$ - среднее значение $f$ по $[a, b]$.


\subsection{Кусочно-непрерывная функция}

$f$ - кусочно-непрерывная функция на $[a, b]$ - ограниченная функция, непрерывная кроме конечного числа точек, в которых у неё разрывы $\MakeUppercase{\romannumeral 1}$ рода.

$f(x) = [x]$, $f(x) = sign(x)$.


\subsection{Почти первообразная}

$f$ - некоторая функция на $[a, b]$, $F : [a, b] \longrightarrow \R$ - почти первообразная $f$, если:

1) $F'(x) = f(x)$  $\forall x \in [a, b]$ кроме конечного числа точек.

2) $F$ - непрерывна.


\subsection{Функция промежутка, аддитивная функция промежутка}

$Segm<a,b> =$ множество всевозможных отрезков лежащих в $<a, b>$.

Функция промежутка Ф : $Segm<a,b> \longrightarrow \R$.

Аддитивная функция промежутка Ф:

$\forall [p, q] \in Segm<a, b>$ $\forall c \in (p, q)$: Ф$([p, q]) =$ Ф$([p, c])$ + Ф$([c, q])$.
\newpage

\end{document}